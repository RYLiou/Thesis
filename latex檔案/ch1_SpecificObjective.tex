%\input{preamble1}
%\usepackage{wallpaper}                                          % 使用浮水印
%\CenterWallPaper{0.6}{images/ntpu.eps}                           % 浮水印圖檔
%\begin{document}
%\fontsize{12}{22pt}\selectfont
\cleardoublepage
\thispagestyle{empty}
\setlength{\parindent}{2em}
\chapter{緒論}

\section{研究動機}
		
	現今大多預測任務依資料型態可分為回歸(Regression)與分類(Classification)兩種,資料型態分別為連續型變數(Numerical)與類別型變數(Categorical),類別型變數又可分為名目型資料(Nominal)與次序型資料(Ordinal),在次序型資料中,各類別間彼此距離不完全相同且存在大小關係,而在名目型資料中,各類別間彼此視為等距且不存在大小關係,分類模型討論的是類別型目標變數的問題。

	常見的分類模型為二元及多元邏輯斯模型(Logit Model)是處理名目型目標變數之分類模型,McCullagh(1980)提出次序型目標變數之分類模型,該模型是邏輯斯模型之推廣,可以稱為次序邏輯斯模型(Ordered Logit Model)。
	
	Frank與Hall(2001)和Cardoso與Pinto da Costa(2007)皆提到,現今機器學習的方法中,遇上多分類的問題通常都假設預測變數為名目型資料,即沒有考慮到目標變數本身大小順序的關係,次序型資料的預測方法還是沒有受到太大的重視。次序型資料在醫學臨床研究與社會科學領域很常見,在許多實際案例中也很常能見到次序型的資料,例如醫學中輔助治療裝置的情況(從無症狀且能自理至死亡分為7個等級)、債券評級(非常同意、同意、普通、不同意與非常不同意)、政府估計國家支出水平(高、中與低)與就業狀態(未就業、兼職與完全就業)等;若目標變數本身為連續型變數,例如溫度,我們也可將其分組並定義為熱、正常與冷三類。
	
	總的來說,次序型資料相較名目型資料更可以展現出資料本身正確的大小關係與順序性,而我們在做分析時也應該將資料變數本身順序關係一併考慮進去。近年來由於資訊科技蓬勃發展,深度學習與自然語言處理也隨之興起,網路電影評論、飯店評論與Google評論都會有一般民眾的留言與評論等級,通常分為1到5分,而過去在文字語意預測方面的研究大都還是將目標變數視為名目型,在此我們也將嘗試實作文字資料於次序型分類模型的預測,若分類模型能透過文字資料成功預測該評論對應之評分,則未來在面對僅有文字,沒有對應評分的評論,將可以使用分類模型有效的判斷文字評論的正、負面意見等級。
	
\section{研究目的}
	
	本研究使用的模型分為次序型分類模型與名目型分類模型,次序型分類模型包含三種不同的次序邏輯斯模型(Ordered Logit Model),分別為Cumulative Logit Model, Continuation-Ratio Logit Model與Adjacent-Category Logit Model,名目型分類模型包含樸素貝葉斯(Naïve Bayes)與多元邏輯斯模型(Multinomial Logistic Model),將次序型分類模型與名目型分類模型使用在多組目標變數為次序型尺度的資料集比較預測結果。衡量指標使用正確率(Accuracy)、Macro-F1與均方誤差(MSE),其中Macro-F1將會平衡並考慮到召回率(Recall)與準確率(Precision),均方誤差雖然不適用於類別型資料,但這指標可以協助我們看出預測的遠近,例如分數有1到5分,若實際分數為1分時預測成2分和5分雖然都預測錯誤,但因資料本身具有大小關係,預測成2分會比預測成5分更接近實際分數。
	
	次序邏輯斯模型在符合比例賠率假設(Proportional Odds Assumption)之下有更好的模型解釋能力,我們使用統計模擬的方式產生符合比例賠率假設(Proportional Odds Assumption)之下的資料,驗證是否資料在符合假設之下用於次序邏輯斯模型會有更好的預測結果,因為雖然資料定義為次序型資料,但若其分佈並不具有次序型資料的分佈樣態,代表次序型資料與名目型資料無異。此外因現實中經常遇到不平衡資料,我們也會比較不平衡資料的重抽樣方法是否助於最後預測結果。
	
	最後我們將實作文字資料於名目型與次序型分類模型的比較,資料集有中文評論資料與英文評論資料各一份,分別為Yahoo電影網站上2020年上映的每部國片中的每則評論與對應評分與Kaggle平台上提供的Trip Advisor Hotel Reviews,其中評分分數為1至5分,可以視為次序型資料,我們會從文字評論中萃取出特徵(Features),將評分分數作為目標變數,比較次序型與名目型資料的分類模型預測之表現,過程將會包含文字資料的預處理與詞嵌入(Word Embedding),其中詞嵌入在文字資料轉換至預測模型的過程極為重要,在此我們將使用四種詞嵌入的方法並檢視最終結果,分別為Wiki Pretrain Word Embedding、TF-IDF、CBOW與Skip-gram。

\newpage
%\end{document

