%\documentclass[12pt, oneside, a4paper]{book}
\documentclass[12pt, a4paper]{book}
%----- 定義使用的 packages ----------------------
\usepackage{subcaption,booktabs,float}
\usepackage{arydshln}
\usepackage{algorithm}
%\usepackage{algorithmic}
\usepackage{algpseudocode}
\usepackage{amsmath}

\usepackage{amsmath}
\usepackage{amssymb}

\usepackage{threeparttable}

\usepackage{fontspec} 						% Font selection for XeLaTeX;  
\usepackage{xeCJK}							% 中文使用 XeCJK,但利用 \setCJKmainfont 定義粗體與斜體的字型
\defaultfontfeatures{Mapping=tex-text} 		% to support TeX conventions like ``---''
\usepackage{xunicode} 						% Unicode support for LaTeX character names (accents, European chars, etc)
\usepackage{xltxtra} 						% Extra customizations for XeLaTeX
\usepackage[sf,small]{titlesec}
\usepackage{amsmath, amssymb}
\usepackage{amsthm}							% theroemstyle 需要使用的套件
\usepackage{bm}                             % 排版粗體數學符號
\usepackage{enumerate}
%\usepackage{graphicx, subfig, float} 		% support the \includegraphics command and options
\usepackage{array}
\usepackage{color, xcolor}
\usepackage{longtable, lscape}              % 跨頁的超長表格;lscape是旋轉此類表格的
\usepackage{threeparttable}                 % 巨集,使表格加註解更容易(手冊p169)
\usepackage{multirow, booktabs}             % 讓表格編起來更美的套件(手冊p166),編輯跨列標題重覆的表格(手冊p182)
\usepackage{colortbl}   

\usepackage[titletoc]{appendix}

\usepackage{pdfpages} %插入pdf

%.............................................表格標題註解之巨集套件
\usepackage{natbib}							% for Reference
\usepackage{makeidx}						% for Indexing
\usepackage[parfill]{parskip} % Activate to begin paragraphs with an empty line rather than an indent
%\usepackage{geometry} % See geometry.pdf to learn the layout options. There are lots.
\usepackage{geometry}
\geometry{a4paper,%
  paperwidth=21cm,paperheight=29.6cm,%
  top=2.3cm,bottom=3.5cm,left=3.0cm,right=2.5cm,%
  %showframe,%
  nohead%
} % End of \geometry{}

% API用來 '開始使用' 和 '停止使用' 封面邊界
\newcommand{\EnableCoverPageStyle}{%
  \newgeometry{%
    top=2.3cm,bottom=3cm,left=2.0cm,right=1cm,%
    nohead,nofoot%
  }%
} % End of \newcommand{}

\newcommand{\DisableCoverPageStyle}{\restoregeometry}
%\usepackage[left=1.5in,right=1in,top=1in,bottom=1in]{geometry} 
%\usepackage[left=1.5in,right=1.5in,top=1in,bottom=1in]{geometry} 
\usepackage{framed} 
\usepackage{titlesec}    
\usepackage{indentfirst}
\setlength{\parindent}{0em} %%控幾格
\usepackage{url}                            % 文稿內徵引網址
    \def\UrlFont{\rm}                       % 網頁
\usepackage{fancyhdr}
	\pagestyle{fancy}
	\fancyhf{}                              % 清除所有頁眉頁足
	\renewcommand{\headrulewidth}{0pt}      % 頁眉下方的橫線    
%-----------------------------------------------------------------------------------------------------------------------
%  主字型設定
\setCJKmainfont
	[AutoFakeBold=true,AutoFakeSlant=true]
%	{cwTeX Q Ming Medium} 					% 設定中文內文字型
	{標楷體}	
\setmainfont{Times New Roman}				% 設定英文內文字型
\setsansfont{Arial}							% used with {\sffamily ...}
%\setsansfont[Scale=MatchLowercase,Mapping=tex-text]{Gill Sans}
\setmonofont{Courier New}					% used with {\ttfamily ...}
%\setmonofont[Scale=MatchLowercase]{Andale Mono}
% 其他字型(隨使用的電腦安裝的字型不同,用註解的方式調整(打開或關閉))
% 英文字型
\newfontfamily{\R}{Imprint MT Shadow}
\newfontfamily{\E}{Cambria}
\newfontfamily{\A}{Arial}
\newfontfamily{\C}[Scale=0.9]{Cambria}
\newfontfamily{\TT}[Scale=0.8]{Times New Roman}
% 中文字型
\newCJKfontfamily{\MB}{微軟正黑體}			% 適用在 Mac 與 Win
\newCJKfontfamily{\SM}[Scale=0.8]{新細明體}	% 縮小版
\newCJKfontfamily{\K}{標楷體}                % Windows 下的標楷體
%\newCJKfontfamily{\K}{Kaiti TC Regular}    % Mac OS 下的標楷體
%\newCJKfontfamily{\BM}{Heiti TC Medium}	% Mac OS 下的黑體(粗體)
%\newCJKfontfamily{\SR}{Songti TC Regular}	% Mac OS 下的宋體
%\newCJKfontfamily{\SB}{Songti TC Bold}		% Mac OS 下的宋體(粗體)
%\newCJKfontfamily{\CF}{cwTeX Q Fangsong Medium}	% CwTex 仿宋體
%\newCJKfontfamily{\CB}{cwTeX Q Hei Bold}			% CwTex 粗黑體
%\newCJKfontfamily{\CK}{cwTeX Q Kai Medium}   		% CwTex 楷體
%\newCJKfontfamily{\CM}{cwTeX Q Ming Medium}		% CwTex 明體
%\newCJKfontfamily{\CR}{cwTeX Q Yuan Medium}		% CwTex 圓體
%-----------------------------------------------------------------------------------------------------------------------
\XeTeXlinebreaklocale "zh"                  		%這兩行一定要加,中文才能自動換行
\XeTeXlinebreakskip = 0pt plus 1pt     
%-----------------------------------------------------------------------------------------------------------------------
%----- 重新定義的指令 ---------------------------
\newcommand{\cw}{\texttt{cw}\kern-.6pt\TeX}			% 這是 cwTex 的 logo 文字
\newcommand{\imgdir}{images/}						% 設定圖檔的位置
\renewcommand{\tablename}{表}						% 改變表格標號文字為中文的「表」(預設為 Table)
\renewcommand{\figurename}{圖}						% 改變圖片標號文字為中文的「圖」(預設為 Figure)
\renewcommand{\contentsname}{目~錄}
\renewcommand\listfigurename{圖目錄}
\renewcommand\listtablename{表目錄}
\renewcommand{\appendixname}{附~錄}                  
\renewcommand{\indexname}{索引}
\renewcommand{\bibname}{參考文獻}
%-----------------------------------------------------------------------------------------------------------------------

\theoremstyle{plain}
\newtheorem{de}{Definition}[section]				%definition獨立編號
\newtheorem{thm}{定理}[section]						%theorem 獨立編號,取中文名稱並給予不同字型
\newtheorem{lemma}[thm]{引理}						%lemma 與 theorem 共用編號
\newtheorem{ex}{{\E Example}}						%example 獨立編號,不編入小節數字,走流水號。也換個字型。
\newtheorem{cor}{Corollary}[section]				%not used here
\newtheorem{exercise}{EXERCISE}						%not used here
\newtheorem{re}{\emph{Result}}[section]				%not used here
\newtheorem{axiom}{AXIOM}							%not used here
\renewcommand{\proofname}{\textbf{Proof}}			%not used here

\newcommand{\loflabel}{圖} 							% 圖目錄出現 圖 x.x 的「圖」字
\newcommand{\lotlabel}{表}  							% 表目錄出現 表 x.x 的「表」字

\parindent=0pt

%--- 去除圖表冒號 ---
\usepackage{caption}
\DeclareCaptionLabelSeparator{twospace}{\ ~}
\captionsetup{labelsep=twospace}
%------------------

%--- 其他定義 ----------------------------------
% 定義章節標題的字型、大小
\titleformat{\chapter}[display]{\centering\LARGE\K\bf}		% 定義章抬頭靠右(\reggedleft)
 { 第\ \thechapter\ 章}{0.2cm}{}
%\titleformat{\chapter}[hang]{\centering\LARGE\sf}{\MB 第~\thesection~章}{0.2cm}{}%控制章的字體
%\titleformat{\section}[hang]{\Large\sf}{\MB 第~\thesection~節}{0.2cm}{}%控制章的字體
%\titleformat{\subsection}[hang]{\centering\Large\sf}{\MB 第~\thesubsection~節}{0.2cm}{}%控制節的字體
\titleformat*{\section}{\normalfont\Large\bf}
\titleformat*{\subsection}{\normalfont\large\bf}
\titleformat*{\subsubsection}{\normalfont\bf}


% 顏色定義
\definecolor{heavy}{gray}{.9}						% 0.9深淺度之灰色
\definecolor{light}{gray}{.8}
\definecolor{pink}{rgb}{0.99,0.91,0.95}             % 定義pink顏色
