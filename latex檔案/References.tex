%\input{preamble1}
%\usepackage{wallpaper}                                          % 使用浮水印
%\CenterWallPaper{0.6}{images/ntpu.eps}                           % 浮水印圖檔
%\begin{document}
\fontsize{12}{22pt}\selectfont
\cleardoublepage
\thispagestyle{empty}
\setlength{\parindent}{2em}
\chapter*{參考文獻}

\noindent Alan Agresti(2003). \textit{Categorical Data Analysis 3rd Edition}, A JOHN WILEY $\&$ SONS, INC., PUBLICATION.

\noindent Jones, K. S.(1972). A statistical interpretation of term specificity and its application in retrieval. \textit{Journal of documentation}.

\noindent Liu, B.(2020). Text sentiment analysis based on CBOW model and deep learning in big data environment. \textit{Journal of Ambient Intelligence and Humanized Computing}, $\bm{11(2)}$, 451-458.

\noindent McCullagh, P.(1980). Regression models for ordinal data.\textit{Journal of the Royal Statistical Society: Series B (Methodological)}, $\bm{42(2)}$, 109-127.

\noindent Cardoso, J., $\&$ da Costa, J. P.(2007). Learning to Classify Ordinal Data: The Data Replication Method. \textit{Journal of Machine Learning Research}, $\bm{8}$, 1393-1429.

\noindent Chu, W., $\&$ Keerthi, S. S.(2005). New approaches to support vector ordinal regression. \textit{In Proceedings of the 22nd international conference on Machine learning}, 145-152.

\noindent Frank, E., $\&$ Hall, M.(2001). A simple approach to ordinal classification. \textit{ECML'01: Proceedings of the 12th European Conference on Machine Learning}, 145-156.

\noindent Jain, A. P., $\&$ Dandannavar, P.(2016). Application of machine learning techniques to sentiment analysis. \textit{In 2016 2nd International Conference on Applied and Theoretical Computing and Communication Technology (iCATccT)}, 628-632.

\noindent Koren, Y., $\&$ Sill, J.(2011). Ordrec: an ordinal model for predicting personalized item rating distributions. \textit{In Proceedings of the fifth ACM conference on Recommender systems}, 117-124.


\newpage 


\noindent Opitz, J., $\&$ Burst, S.(2019). Macro f1 and macro f1. \textit{arXiv preprint arXiv:1911.03347}.

\noindent Rennie, J. D., $\&$ Srebro, N.(2005). Loss functions for preference levels: Regression with discrete ordered labels. \textit{In Proceedings of the IJCAI multidisciplinary workshop on advances in preference handling}, $\bm{1}$.

\noindent Saad, S. E., $\&$ Yang, J.(2019). Twitter sentiment analysis based on ordinal regression. \textit{IEEE Access}, $\bm{7}$, 163677-163685.

\noindent Jing, L. P., Huang, H. K., $\&$ Shi, H. B.(2002). Improved feature selection approach TFIDF in text mining. \textit{In Proceedings. International Conference on Machine Learning and Cybernetics, $\bm{2}$}, 944-946.

\noindent Joulin, A., Grave, E., $\&$ Dandannavar, P.(2016). Bag of tricks for efficient text classification. \textit{arXiv preprint arXiv:1607.01759}.

\noindent Vargas, V. M., Gutiérrez, P. A., $\&$  Hervás-Martínez, C.(2020). Cumulative link models for deep ordinal classification. \textit{Neurocomputing, }$\bm{401}$, 48-58.

\noindent Bojanowski, P., Grave, E., Joulin, A., $\&$ Mikolov, T.(2017). Enriching word vectors with subword information. \textit{Transactions of the Association for Computational Linguistics}, $\bm{5}$, 628-632.

\noindent Liu, C., Li, Y., Ping Li, $\&$  Fei, H.(2019). Deep Skip-Gram Networks for Text Classification. \textit{In Proceedings of the 2019 SIAM International Conference on Data Mining}, 145-153.

\noindent Mikolov, T., Sutskever, I., Chen, K., Corrado, G., $\&$ Dean, J.(2013). Distributed representations of words and phrases and their compositionality. \textit{arXiv preprint arXiv:1310.4546}.





\newpage
\thispagestyle{empty}


%\end{document}