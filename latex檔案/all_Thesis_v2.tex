\input{preamble}
\usepackage{wallpaper}                                          % 使用浮水印
\CenterWallPaper{0.6}{images/ntpu.eps}                           % 浮水印圖檔	
%--- 製作封面頁 ----------------------------------
\begin{document}

\fontsize{12}{22pt}\selectfont
\thispagestyle{empty}
\vspace*{1cm}
 \begin{center}
    \huge 國~立~政~治~大~學~統~計~系~研~究~所 \\ 碩~士~學~位~論~文
 \end{center}
\vspace*{2cm}

\vspace*{3cm}
 \begin{center}
    \LARGE 名目型與次序型資料之分類模型比較及其在網路文本評論之應用
 \end{center}

 \begin{center}    			   
      \Large A comparison of nominal and ordinal classification models with application to online reviews
 \end{center}
\vspace*{6cm}
 \begin{center}
    \LARGE 指導教授:翁久幸~~博士 \\
 \end{center}
\begin{center}
    \LARGE 研究生:柳瑞俞~~撰\\
\vspace*{2cm}
    \LARGE 中華民國~一一$\bigcirc$~年~七~月
\end{center}

\newpage
\cleardoublepage

\setcounter{tocdepth}{2}		% 目錄層次
\pagenumbering{roman}			% 羅馬文頁碼
%\include{Thanks}   		% 謝誌
%\documentclass[12pt]{article}
%\usepackage{fancyhdr}
%\usepackage{indentfirst}
%
%\begin{document}
%\fontsize{12}{20pt}\selectfont

%\fancyhf{}
%%%%%%%%%%%%%%%%%%%%%%%%%%% 謝辭 %%%%%%%%%%%%%%%%%%%%%%%%%%%%%%%%%%%%%%%%%%%%%

\newpage
\cleardoublepage
%\fontsize{12}{18pt}\selectfont

\setlength{\parindent}{2em}
\thispagestyle{empty}

\vspace*{1cm}

\begin{center}
{\Large \bf 致謝}\\[20pt]     
\end{center}


	時光飛逝,一轉眼兩年在政大的求學階段即將結束,回想起自己讀研究所的這兩年,覺得自己非常的幸運,遇到了一群很好的同學與教授,首先要感謝指導教授\;翁久幸\;老師的指導,一開始給我許多空間提出自己的想法,在討論的過程指點出我的問題並給予我建議,讓我的研究一步一步有了更明確的方向,老師在我的實驗過程與整個論文邏輯上都給了我極大的幫助,也非常感謝老師在我低潮的時候給予我很多鼓勵,不僅僅只是在論文上的幫助。感謝系上給予我幫助過的每位教授,學生畢業後不會忘記老師們給過的恩惠。
	
	感謝在這兩年遇到的同學們,與你們一起待在研究室的日子非常開心,不論是大家一起討論報告、做作業,或是一起玩遊戲放鬆,都替我研究所階段增添不少色彩。也很感謝這兩年幫助我的朋友們,在我低潮的時候一句暖心的問候給我的幫助都非常大,謝謝大家的互相扶持。

	謝謝我的家人,在我求學階段不曾給過我壓力,提供一個良好的環境讓我無後顧之憂的唸書。回想起自己讀研究所的動機一路到研究所畢業,至今還是覺得不可思議,我真的非常幸運才能夠順利的一路到這,謝謝這一路上遇到的大家。


\vspace*{2cm}


\begin{flushright} 
{柳瑞俞\\ 國立政治大學統計學系\\ July 2021\\}
\end{flushright} 

\vspace*{1cm}


%%%%%%%%%%%%%%%%%%%%%%%%%%% 摘要 %%%%%%%%%%%%%%%%%%%%%%%%%%%%%%%%%%%%%%%%%%%

\newpage 
\cleardoublepage 
%\fontsize{12}{18pt}\selectfont  
\setlength{\parindent}{2em} 
\thispagestyle{empty}  
\vspace*{1cm}  

\begin{center} 
{\Large \bf 摘要}\\[20pt]      
\end{center}

   	隨著資訊科技的蓬勃發展,機器學習的技術越來越被大眾所使用,然而現今面對次序型的資料型態多半直接使用名目型分類模型而不是使用能夠正確考慮資料本身大小關係的次序型分類模型,McCullagh(1980)提出次序型目標變數的邏輯斯模型之推廣,稱為次序邏輯斯模型(Ordered Logit Model),本研究使用三種次序邏輯斯模型做為次序型分類模型,在名目型分類模型的部分使用樸素貝葉斯(Naïve Bayes)與多元邏輯斯模型,用來預測13組目標變數為次序型的資料集,並以正確率(Accuracy)、Macro-F1與均方誤差(MSE)做為衡量指標,結果發現只有其中六組資料集在次序型分類模型表現較好,進而我們發現這六組資料集中較多變數符合次序邏輯斯模型的「比例賠率假設(Proportional odds assumption)」,接著我們使用統計資料模擬的方法,驗證確實在符合模型假設之下的資料,使用次序型分類模型獲得較名目型分類模型佳的預測結果。
   	   	
   	最後我們將次序型資料的問題延伸至現今流行的文字分類議題,電影與Google評論等都會有一般民眾的留言與評論等級,通常分為1到5分,我們使用Word2Vec、TF-IDF與Fasttext的詞嵌入(Word Embedding)方式將文字資料轉為模型可以代入的向量型態,結果顯示中文評論使用次序型分類模型成效較佳,英文評論使用名目型分類模型較佳,詞嵌入方法也會影響預測結果,考慮越多周遭字詞的Word2Vec方法成效越好,TF-IDF法表現最差,但Word2Vec訓練方式較久,若有時間上的考量可以使用網路上使用Fasttext訓練好的Wiki Pretrain詞向量也有不差的成效。
   	
   	
 	

\vspace*{1cm}
\noindent {\scshape 關鍵詞}: 次序邏輯斯模型、多元邏輯斯模型、Word2Vec、TF-IDF、FastText

%%%%%%%%%%%%%%%%%%%%%%%%%%% ABSTRACT %%%%%%%%%%%%%%%%%%%%%%%%%%%%%%%%%%%%%%%%%%%%%
\newpage 
\cleardoublepage 
%\fontsize{12}{18pt}\selectfont  
\setlength{\parindent}{2em} 
\thispagestyle{empty}  
\vspace*{1cm} 
\begin{center} 
{\Large \bf Abstract}\\[20pt]      
\end{center}


	With the development of information technology, machine learning techniques are increasingly being used by the public. However, nowadays, when facing ordinal data, most of them use the nominal classification model instead of the ordinal classification that can correctly consider the rank relationship of the data. McCullagh (1980) proposed an extension of the logistic model of ordered target variables, called the ordered logit model. This study uses three ordered logit models as the ordinal classification model. Part of the nominal classification models uses Naïve Bayes and multinomial logit model to predict 13 sets of target variables as ordinal data, and uses Accuracy, Macro-F1 and Mean Square Error (MSE) As a measurement, it turns out that only six datasets perform better in the ordinal classification model. Then we found that more variables in these six datasets conform to the "Proportional odds assumption" of the ordered logistic model. Then we use statistical data simulation methods to verify that the data is indeed in line with the model assumptions, and use the ordinal classification model to obtain better prediction results than the nominal classification model.
	
    Finally, we extend the problem of ordinal data to the text classification issues. Movies and Google reviews will have public comments and ratings. They are usually divided into 1 to 5 points. The word embedding method we use Word2Vec, TF-IDF and FastText to convert the text data into a vector type that the model can use. The results show that the ordinal classification model for Chinese reviews is better , and the nominal classification model for English reviews is better. The word embedding method will also affect the prediction. As a result, the Word2Vec method that considers more surrounding words the better, the TF-IDF method performs the worst, but the training time of Word2Vec is longer, if you have time considerations, you can use the Wiki Pretrain word vector trained on the Internet using Fasttext, and it will have not bad results.

%\vspace*{1cm}
\noindent {\scshape KEY WORDS}: Ordered Logit Model、Multinomial Logit Model、Word2Vec、TF-IDF、FastText

%\end{document}


   	% 中英文摘要

%%%%%%%%%%%%%%%%%%%%%%%%%% 製作目錄 %%%%%%%%%%%%%%%%%%%%%%%%%%%%%%%%%%%%%%
\newpage
\cleardoublepage		
\fontsize{12}{22pt}\selectfont 	% 字型大小與行距視使用的字型調整
\fancyfoot[C]{\thepage}


\tableofcontents
\newpage
\renewcommand{\numberline}[1]{\loflabel~#1\hspace*{1em}}% 圖目錄出現 圖 x.x 的「圖」字
\listoffigures %圖目錄
\newpage
\renewcommand{\numberline}[1]{\lotlabel~#1\hspace*{1em}}% 表目錄出現 表 x.x 的「表」字
\listoftables % 表目錄
\newpage
\thispagestyle{empty}

%%%%%%%%%%%%%%%%%%%%%%%%% 正文開始 %%%%%%%%%%%%%%%%%%%%%%%%%%%%%%%%%%%%%%
\fontsize{12}{22pt}\selectfont
\cleardoublepage
\ifodd \count0 \else \thispagestyle{empty} \mbox{} \clearpage \fi % 如果是偶數頁就呈現空白頁
\pagenumbering{arabic}          % 重新開始計算頁碼-格式阿拉伯文
\setcounter{page}{1}
%\input{preamble1}
%\usepackage{wallpaper}                                          % 使用浮水印
%\CenterWallPaper{0.6}{images/ntpu.eps}                           % 浮水印圖檔
%\begin{document}
%\fontsize{12}{22pt}\selectfont
\cleardoublepage
\thispagestyle{empty}
\setlength{\parindent}{2em}
\chapter{緒論}

\section{研究動機}
		
	現今大多預測任務依資料型態可分為回歸(Regression)與分類(Classification)兩種,資料型態分別為連續型變數(Numerical)與類別型變數(Categorical),類別型變數又可分為名目型資料(Nominal)與次序型資料(Ordinal),在次序型資料中,各類別間彼此距離不完全相同且存在大小關係,而在名目型資料中,各類別間彼此視為等距且不存在大小關係,分類模型討論的是類別型目標變數的問題。

	常見的分類模型為二元及多元邏輯斯模型(Logit Model)是處理名目型目標變數之分類模型,McCullagh(1980)提出次序型目標變數之分類模型,該模型是邏輯斯模型之推廣,可以稱為次序邏輯斯模型(Ordered Logit Model)。
	
	Frank與Hall(2001)和Cardoso與Pinto da Costa(2007)皆提到,現今機器學習的方法中,遇上多分類的問題通常都假設預測變數為名目型資料,即沒有考慮到目標變數本身大小順序的關係,次序型資料的預測方法還是沒有受到太大的重視。次序型資料在醫學臨床研究與社會科學領域很常見,在許多實際案例中也很常能見到次序型的資料,例如醫學中輔助治療裝置的情況(從無症狀且能自理至死亡分為7個等級)、債券評級(非常同意、同意、普通、不同意與非常不同意)、政府估計國家支出水平(高、中與低)與就業狀態(未就業、兼職與完全就業)等;若目標變數本身為連續型變數,例如溫度,我們也可將其分組並定義為熱、正常與冷三類。
	
	總的來說,次序型資料相較名目型資料更可以展現出資料本身正確的大小關係與順序性,而我們在做分析時也應該將資料變數本身順序關係一併考慮進去。近年來由於資訊科技蓬勃發展,深度學習與自然語言處理也隨之興起,網路電影評論、飯店評論與Google評論都會有一般民眾的留言與評論等級,通常分為1到5分,而過去在文字語意預測方面的研究大都還是將目標變數視為名目型,在此我們也將嘗試實作文字資料於次序型分類模型的預測,若分類模型能透過文字資料成功預測該評論對應之評分,則未來在面對僅有文字,沒有對應評分的評論,將可以使用分類模型有效的判斷文字評論的正、負面意見等級。
	
\section{研究目的}
	
	本研究使用的模型分為次序型分類模型與名目型分類模型,次序型分類模型包含三種不同的次序邏輯斯模型(Ordered Logit Model),分別為Cumulative Logit Model, Continuation-Ratio Logit Model與Adjacent-Category Logit Model,名目型分類模型包含樸素貝葉斯(Naïve Bayes)與多元邏輯斯模型(Multinomial Logistic Model),將次序型分類模型與名目型分類模型使用在多組目標變數為次序型尺度的資料集比較預測結果。衡量指標使用正確率(Accuracy)、Macro-F1與均方誤差(MSE),其中Macro-F1將會平衡並考慮到召回率(Recall)與準確率(Precision),均方誤差雖然不適用於類別型資料,但這指標可以協助我們看出預測的遠近,例如分數有1到5分,若實際分數為1分時預測成2分和5分雖然都預測錯誤,但因資料本身具有大小關係,預測成2分會比預測成5分更接近實際分數。
	
	次序邏輯斯模型在符合比例賠率假設(Proportional Odds Assumption)之下有更好的模型解釋能力,我們使用統計模擬的方式產生符合比例賠率假設(Proportional Odds Assumption)之下的資料,驗證是否資料在符合假設之下用於次序邏輯斯模型會有更好的預測結果,因為雖然資料定義為次序型資料,但若其分佈並不具有次序型資料的分佈樣態,代表次序型資料與名目型資料無異。此外因現實中經常遇到不平衡資料,我們也會比較不平衡資料的重抽樣方法是否助於最後預測結果。
	
	最後我們將實作文字資料於名目型與次序型分類模型的比較,資料集有中文評論資料與英文評論資料各一份,分別為Yahoo電影網站上2020年上映的每部國片中的每則評論與對應評分與Kaggle平台上提供的Trip Advisor Hotel Reviews,其中評分分數為1至5分,可以視為次序型資料,我們會從文字評論中萃取出特徵(Features),將評分分數作為目標變數,比較次序型與名目型資料的分類模型預測之表現,過程將會包含文字資料的預處理與詞嵌入(Word Embedding),其中詞嵌入在文字資料轉換至預測模型的過程極為重要,在此我們將使用四種詞嵌入的方法並檢視最終結果,分別為Wiki Pretrain Word Embedding、TF-IDF、CBOW與Skip-gram。

\newpage
%\end{document


%\input{preamble1}
%\usepackage{wallpaper}                                          % 使用浮水印
%\CenterWallPaper{0.6}{images/ntpu.eps}                           % 浮水印圖檔
%\begin{document}
%\fontsize{12}{22pt}\selectfont
\cleardoublepage
\thispagestyle{empty}
\setlength{\parindent}{2em}
\chapter{文獻回顧}

  	Frank與Hall(2001)提出一種在分類模型上事先處理次序型資料的方法,將k個目標變數類別轉換為一系列k-1次的二分類問題,並使用條件機率的方式考慮預測該類別的機率,如$P(Y > Cool|X)$、$P(Y > Mild|X)$與$P(Y = Hot|X) = P(Y > Cool|X) - P(Y > Mild|X)$,這使得一般分類算法能夠包含到次序資料本身的特性,最後透過該方法與決策樹的結合表現優於分類模型的樸素貝葉斯(Naïve Bayes)。
  	
  	
	 Cardoso與Pinto da Costa(2007)提出使用數據複製方法(Data Replication Method)將次序型資料問題轉換為二分類問題來處理,該方法在k個目標變數類別問題時做了k-1次二分類預測,每次分類後會給予兩邊所對應的權重,第一次分類會對於$\{C_1\}$與$\{C_2, C_3\}$進行二分類預測並給予兩邊權重,第二次分類針對$\{C_1, C_2\}$與$\{C_3, C_4\}$進行二分類預測並給予兩邊權重,以此概念將k-1次分類做完並依據被給予的權重做出最後的分類,文獻中提到其研究著重於機器學習方法的應用,特別是神經網絡(Neural Network)和支持向量機(SVM)在次序資料分類問題上的應用,且該方法在最後資料實證優於傳統分類問題所用的支持向量機模型(SVM)。Vargas等(2020)提出了一種用於順序回歸的深層卷積神經網絡模型(Deep Convolutional Neural Network),在神經網絡的輸出層考慮Cumulative link models不同的鏈結函數。
	 
	 上述了解到對於次序型資料的問題,現今機器學習的方法遇上多分類的問題通常都是假設預測變數為名目型資料,即沒有考慮到目標變數本身大小順序的關係,次序型資料的預測方法還是沒有受到太大的重視,而在提出針對次序型資料的處理方法或是在模型預測時將分類問題做為次序問題來考慮,其表現會優於一般的分類模型,代表考慮正確的資料本質有助於最後的模型預測。
	 
	 在文字資料的部分,因文字是無法直接一個字一個字去代入模型的,不管是在中文或英文文字上我們都會先將文字轉為向量,以一連串的向量代表一段文字, Liu(2020)與Liu等(2019)分別提到在做文本分類的命題時使用CBOW法與Skip - gram法能得捕捉到文本的局部特徵,在做詞嵌入(Word Embedding)時使用這兩種方法能得到良好的結果,而Liu等(2019)更是提出一種基於Skip - gram法的循環神經網絡(Recurrent Neural Network)將文字轉換為向量時獲得更準確的信息;Jing等(2002)則是以傳統TF-IDF法做為處理文字的方法,並以基於TF-IDF的特徵選取法對文字預處理做出改善,而後做為文字的詞嵌入,由上述我們可以知道在文字詞嵌入上有非常多種作法,且對於文字預處理的部分是會顯著的影響最終成效。
	 
	 Elbagir與Yang(2019)和Dandannavar與Jain(2016)對Twitter上的文字評論做情感分析,兩者分別以將目標變數視為五類與三類,分別為非常正向、正向、中立、負面、非常負面與正面、中立、負面這類型的次序型資料去做預測,而兩篇研究皆使用機器學習的方法去做預測,如支持向量回歸(SVR)、決策樹(Decision Tree)和隨機森林(Random Forest)等,前者作者以提出的推文極性分數去定義一則留言的五個等級,再去做預測;後者則是強調在文字特徵提取的部分,包含詞性標註、意見詞、推特的特定功能與否定詞等,上述皆顯示出文字的處理與將次序資料本身特性一併考慮進去對預測的重要性,尤其在網友和客戶留言的平台上更是容易見到此類型的資料。





%\notinput{preamble1}
%\usepackage{wallpaper}                                          % 使用浮水印
%\CenterWallPaper{0.6}{images/ntpu.eps}                           % 浮水印圖檔
%\begin{document}
%\fontsize{12}{22pt}\selectfont
\cleardoublepage
\thispagestyle{empty}
\setlength{\parindent}{2em}
\chapter{研究方法}
	本章節分為四個小節,第一節先介紹衡量模型的指標,透過這三種指標可以從不同的角度去檢視模型預測結果的好壞,第二節介紹三種次序邏輯斯模型(Ordered Logit Model)和兩種名目型分類模型的理論與算法,比較當目標變數為次序型時使用次序型模型去預測和將目標變數視為名目型而使用名目型分類模型去預測的差異,第三節介紹比例賠率假設(Proportional odds assumption)的檢定方法與本研究中模擬資料生成的方法,最後將介紹四種用於文字資料應用時的詞嵌入方法。
	
\section{衡量指標}
	評斷一個模型的配適狀況有多種指標,以下我們選擇三項指標作為評斷模型預測結果的指標,分別為正確率(Accuracy)、Macro-F1與均方誤差(MSE),正確率用來計算預測正確的比率,Macro-F1會平衡考慮到召回率(Recall)與準確率(Precision)的狀況,均方誤差則可以讓我們看出預測值與實際值差異的大小,因為次序型資料本身具有大小關係,均方誤差能表現出此涵義。

\subsubsection{正確率}
	
	我們將以正確率指標來判斷在多分類的情況中,預測結果與實際值相符的狀況,其計算公式如式\ref{eq.1.1.1}與\ref{eq.1.1.2},檢視每筆預測值是否與實際值相同,若相同則代表預測正確,反之預測錯誤則不列入計算。

\begin{equation}\label{eq.1.1.1}
\text{I} = \left\{
\begin{aligned}
\text{1}\;\;\; , \;\;\;Predict = Truth \\
\text{0}\;\;\; , \;\;\;Predict \neq Truth
\end{aligned}
\right.
\end{equation}

\begin{equation}\label{eq.1.1.2}
\text{Accuracy} = \frac{\sum_{i=1}^{n}I_i}{n} \;\;\;;\;\;\;  i=1,...,n
\end{equation}

\subsubsection{Macro-F1}

	F1-Score是透過召回率與準確率計算,以下我們將以圖\ref{grap.3.1.1}混淆矩陣來說明各項公式計算方式,其中召回率與準確率計算公式如式\ref{eq.1.2.1}。
	
	\begin{figure}[H]
    \centering
        \includegraphics[scale=0.7]{\imgdir confusion_matrix.png}
    \caption{混淆矩陣 Confusion Matrix}
    \label{grap.3.1.1}
    
\end{figure}
\begin{equation}\label{eq.1.2.1}
\begin{aligned}
&\text{Precision} = \frac{TP}{TP+FP}  \\
&\text{Recall} = \frac{TP}{TP+FN}
\end{aligned}
\end{equation}

	而F1-Score計算公式如式\ref{eq.1.2.2}
\begin{equation}\label{eq.1.2.2}
	F1 = \frac{2*Precision*Recall}{Precision+Recall} = \frac{2*TP}{2*TP+FP+FN}
\end{equation}

	而在多分類狀況中,每個類別都能計算出各自的F1-Score,Burst與Opitz(2019)指出Macro-F1有兩種不同計算方式,分別為Averaged-F1和F1 of averages:,在此我們選用作者結論所說較為穩健的F1 of averages來計算,其公式如式\ref{eq.1.2.3}。

\begin{equation}\label{eq.1.2.3}
Macro-F1 =\frac{1}{m}\sum_{i=1}^{m} F1_i \;\;\;\; ; i=1,...,m\;\;\;\;\text{m為類別數}
\end{equation}
	
	然而有時候會遇到某類別完全沒有被預測出來的情況,此時準確率分母為零將無法計算,在計算Macro-F1時會出現問題,在此若遇到某類別準確率無法計算的狀況發生時,我們將此類別略過,用其餘類別的F1-Score計算整體Macro-F1。


\subsubsection{均方誤差 MSE}

	均方誤差一般用於數值型的資料,一般來說次序型的資料不適用均方誤差,參考Rennie, 與Srebro(2005)使用到平均絕對誤差(MAE),在此我們也考慮均方誤差,均方誤差可以提供另一層涵義,如分數有1到5分,若實際分數為1分時預測成2分和5分雖然都預測錯誤,但因次序型資料本身具有大小關係,預測成2分會比預測成5分更接近實際分數,其公式如式\ref{eq.1.3.1}。
	
\begin{equation}\label{eq.1.3.1}
\text{MSE}=\frac{1}{n}\sum_{i=1}^{n}(truth_i-predict_i)^2
\end{equation}
	
	
\section{分類模型}
	
	本章節將會介紹三種常見的次序型分類模型,分別為Cumulative Logit Model、Continuation-Ratio Logit Model與Adjacent-Category Logit Model,以及兩種名目型分類模型,分別為樸素貝葉斯 Naïve Bayes與多元邏輯斯模型 Multinomial Logistic Model。
	
\subsection{Cumulative Logit Model}

	令$Y_i$表示次序型隨機變數觀察值,每個觀察值都會有其對應的第j個類別,其中j = 1,...,\;J且j $\geq $ 2。
	
\begin{equation}
\label{eq.3-2-1.1}
\begin{aligned}
	\gamma_{ij} = F(\eta_{ij})\;,\;\; \eta_{ij} = \theta_j - x^{T}_i\beta \;,\;\; i = 1,...,n\;,\;\; j=1,...,J-1 
\end{aligned}
\end{equation}

\begin{equation}
\label{eq.3-2-1.2}
\begin{aligned}
	\gamma_{ij} = P(Y_i\leq j) = \pi_{i1} + ... + \pi_{ij}\;\; \text{with} \;\; \sum_{j=1}^{J} \pi_{ij} = 1
\end{aligned}
\end{equation}
	
	此模型公式如式\ref{eq.3-2-1.1}與\ref{eq.3-2-1.2},$\eta_{ij}$代表線性預測值、$\beta$代表斜率參數,而$F^{-1}$是鏈結函數(Link Function),式\ref{eq.3-2-1.2}是累積機率,$\theta_j$為非遞減之閾值,$\nonumber - \infty \equiv \theta_0 \leq  ... \leq  \theta_J \equiv \infty$。
	

	常見的鏈結函數有 logit、probit、log-log與cauchit,其中最常使用的為logit,對應到邏輯斯模型(見章節3.2.5),而在本研究中的Cumulative Logit Model選擇以logit函數做為鏈結函數,如下式\ref{eq.3-2-1.3}。
	
\begin{equation}
\label{eq.3-2-1.3}
\begin{aligned}
	logit(\gamma_{ij}) =log \frac{P(Y_i\leq j)}{1-P(Y_i\leq j)} = log \frac{\pi_{i, 1}+..+\pi_{i, j}}{\pi_{i, j+1}+...+\pi_{i, J}}
\end{aligned}
\end{equation}

	
	次序型分類模型通常有「比例賠率假設(Proportional odds assumption)」,關於此假設的檢定將在3.3.1章節說明,它的意思是任何解釋變量的影響在不同的閾值上都是一致的,無論閾值如何,解釋變量對機率都有相同的影響,在此模型假設下,每個閾值處都有各自的截距項,但只有一個勝算比(Odds Ratio)影響各個解釋變量,是較為簡約且容易解釋的模型,如式\ref{eq.3-2-1.5}。


\begin{equation}
\label{eq.3-2-1.5}
	logit(\gamma_{ij}) = \beta_{0j} + \beta^{'}X \;\;\;\;\;;\;\;\; j = 1,...,J-1
\end{equation}


\subsection{Continuation-Ratio Logit Model}

	Continuation-Ratio Logit Model為Cumulative Logit Model的變形兩者在基本概念上相似,Continuation-Ratio Logit Model有兩種形式,一種是每個類別相較於高類別的log odds如式\ref{eq.3-2-2.1},另一種則是每個類別相較於較低類別的log odds如式\ref{eq.3-2-2.2}。

\begin{equation}
\label{eq.3-2-2.1}
	log(\frac{\pi_{j}}{\pi_{j+1} + ... + \pi_{J}}) \;\;\;\;;\;\;\; j = 1 ,..., J - 1
\end{equation}

\begin{equation}
\label{eq.3-2-2.2}
	log(\frac{\pi_{j+1}}{\pi_{1} + ... + \pi_{j}}) \;\;\;\;;\;\;\; j = 1 ,..., J - 1
\end{equation}

	當反應變數的次序類別代表各個階段的發展時,這種序列過程使用第一種形式效果較佳,每個類別都必須通過較低的階段才能進入較高的階段,例如接受特定醫療時間,小於1年、介於1\textasciitilde 3年、介於3\textasciitilde 5年與大於5年後的存活時間,以及教育程度小學、國中、高中、大學以上等,在與更高水平比較後捨棄給定水平的結果;反之若反應變數的次序類別反過來時,使用第二種形式的效果會較佳。

	這樣的Continuation-Ratio Logit Model是基於條件之下的logit機率如下式\ref{eq.3-2-2.3},而在比例賠率假設(Proportional odds assumption)的前提假設之下,我們可以得到一個更為簡約且容易解釋的Continuation-Ratio Logit Model模型如下式\ref{eq.3-2-2.4}。
	
\begin{equation}
\label{eq.3-2-2.3}
	\omega_j = P(Y_i = j | Y_i \geq j) = \frac{\pi_j}{\pi_{j+1} + ... + \pi_{J}} \;\;\;\;;\;\;\; j = 1 ,..., J - 1
\end{equation}

\begin{equation}
\label{eq.3-2-2.4}
	logit(\omega_j(x)) =  \beta_{0j} + \beta^{'}X \;\;\;\;\;;\;\;\; j = 1,...,J-1
\end{equation}
	
	
\subsection{Adjacent-Category Logit Model}

	Adjacent-Category Logit Model為Cumulative Logit Model的變形,兩者在基本概念上相似,其特點在於他是使用相鄰類別做log odds的比較,如式\ref{eq.3-2-3.1}。

\begin{equation}
\label{eq.3-2-3.1}
	logit[P(Y_i = j | Y_i= j\;\;or\;\;Y_i = j + 1)] = log\frac{\pi_j}{\pi_{j+1}} \;\;\;\;;\;\;\; j = 1 ,..., J - 1
\end{equation}

	若符合比例賠率假設可以得到一個更為簡約且容易解釋的模型如式\ref{eq.3-2-3.2}。
	
\begin{equation}
\label{eq.3-2-3.2}
	log\frac{\pi_j}{\pi_{j+1}} = \beta_{0j} + \beta^{'}X \;\;\;\;;\;\;\; j = 1 ,..., J - 1
\end{equation}	
	

	當Adjacent-Category Logit Model以同個k為基準類別時表示如式\ref{eq.3-2-3.4}。在這種情況下若將反應變數當作名目尺度來處理,意即不考慮目標變數次序性,則可以將General Adjacent-Category Logit Model表示為Baseline-Category Logit Models,即為多元邏輯斯模型(Multinomial Logistic Model),將於3.2.6節介紹,式\ref{eq.3-2-3.3}為該模型公式。
	
\begin{equation}
\label{eq.3-2-3.4}
log\frac{\pi_1}{\pi_k},\;log\frac{\pi_2}{\pi_k},\;...\;,\;log\frac{\pi_j-1}{\pi_k}
\end{equation}	
	
\begin{equation}
\label{eq.3-2-3.3}
	log\frac{\pi_j}{\pi_{k}} = \beta_{0j} + \beta^{'}_jX \;\;\;\;;\;\;\; j = 1 ,..., J - 1
\end{equation}		
	
	
\subsection{樸素貝葉斯 Naïve Bayes}

	樸素貝葉斯(Naïve Bayes)是一種運用貝氏定理(Bayes Theorem)最大化後驗機率的簡單分類器,經常被用於文字分類,令X代表所有自變數$X_i$的序列,y為目標變數,在計算條件機率$P(X|y)$時,若假設在給定y之下,變數與變數之間獨立,可以得到條件機率的計算方式如下式\ref{eq.3-2-4.1},相較於$P(X_1,X_2,...,X_n|y)$,$P(X_i|y)$更為容易計算。

\begin{equation}
\label{eq.3-2-4.1}
\begin{aligned}
	P(X|y) &= P(x_1,x_2,x_3,...,x_n|y) \\&= P(x_1|y)P(x_2|y)P(x_3|y)...P(x_n|y) \\&= \prod_{i=1}^{n}P(x_i|y)
\end{aligned}
\end{equation}

	隨後我們便可計算各類別最大化後驗機率,並以此做為目標變數Y 所屬類別的判斷依據,如下式\ref{eq.3-2-4.2},此分類器相較於其他 SVM、Random Forest等分類器,優點為計算時間快。
	
\begin{equation}
\label{eq.3-2-4.2}
\begin{aligned}
	&P(y|X) = P(y)\prod_{i=1}^{m}\frac{P(x_i|y)}{P(x_i)} \\&y^{'} = \arg \max\limits_{y\in C}P(y)\prod_{i=1}^{m}\frac{P(x_i|y)}{P(x_i)} \\& \;\;\;= \arg \max\limits_{y\in C}\{logP(y) + \sum_{i=1}^{m}logP(x_i|y)\}
\end{aligned}
\end{equation}

\subsection{多元邏輯斯模型 Multinomial Logistic Model}

	邏輯斯模型(Logistic Model)是用於做二分類的模型,模型如式\ref{eq.3-2-5.2}所示。

\begin{equation}
\label{eq.3-2-5.2}
\begin{aligned}
	Y(x)\sim\~Bernoulli(\pi(x))\;\; \text{with}\;\; \log(x) = \frac{\pi(x)}{1-\pi(x)} = \beta_0+\beta_1x_1+...+\beta_px_p 
\end{aligned}
\end{equation}

\begin{equation} 
\nonumber
\begin{aligned}
\text{根據式3.21可以推得}\pi(x)= F(\beta_0+\beta_1x+...+\beta_px_p)\;\;,\text{其中} \;\; F(t)=\frac{e^t}{1+e^t}
\end{aligned} 
\end{equation}


	傳統二元邏輯斯模型的輸出是二分類的,多元邏輯斯模型則是將二元邏輯斯模型推廣至多類問題,即可以具有兩種以上的分類結果。對K個可能的分類結果運行K-1個獨立二元邏輯斯模型,也就是把某一類別當成是主類別(如類別K),而將其餘K-1個類別與主類別分別進行建模,根據式\ref{eq.3-2-5.4}的可得結果式\ref{eq.3-2-5.5}。

\begin{equation}
\label{eq.3-2-5.4}
\begin{aligned}
P(Y_i=K)& = 1-\sum_{k=1}^{K-1}P(Y_i=k) = 1-\sum_{k=1}^{K-1}P(Y_i=K)e^{\beta_kx_i}\\&
=>P(Y_i=K) = \frac{1}{1+\sum_{k=1}^{K-1}e^{\beta_kx_i}}
\end{aligned}
\end{equation}


\begin{equation}
\label{eq.3-2-5.5}
\begin{aligned}
P(Y_i=m) = \frac{e^{\beta_{m}x_i}}{1+\sum_{k=1}^{K-1}e^{\beta_{k}x_i}}\;\;\;;\;\;\;m = 1,..., K-1 
\end{aligned}
\end{equation}



\section{假設檢定與模擬方法}

	過去統計模型強調的重點為「解釋能力」而非預測能力,在符合統計模型的前提假設之下,只能確保其有良好的模型解釋能力而非代表模型有良好的預測能力,本研究將以資料模擬的方法,模擬出符合前提假設之下的資料,驗證其配適出的模型是否也具有較佳的預測能力。	
	
	比例賠率假設(Proportional odds assumption)為次序型分類模型的前提假設,本章節將先介紹此前提假設與假設檢定的方法,再介紹資料模擬的演算法。
	

\subsection{比例賠率假設及檢定}

	在3.2.1節中簡要介紹了「比例賠率假設(Proportional odds assumption)」,或稱「平行賠率假設(Parallel odds assumption)」,此假設意味著觀察到的解釋變量都有相同的勝算比(Odds ratio),每個等級的模型差別只會差在截距項的不同,並共用每個解釋變數的斜率以獲得一個更簡約的模型。
	
	以表\ref{tab.3-4-2.1}呈現,該資料集解釋變數為性別,類別包含Boys和Girls,反應變數為英文考試成績,共有3、4、5、6與7五種等級,透過勝算比(Odds ratio)可以發現Girls的勝算(Odds)皆比Boys還高,因為達到更高水平的Girls比Boys還多,計算兩者勝算法(Odds ratio)會發現落在(1.77, 2.2)的區間範圍內,若我們給予一個共同勝算比(Odds ratio)為2,分別解讀可以得到Girls達到4+等級的勝算(Odds)比Boys達到4+等級的勝算(Odds)高2倍,Girls達到5+等級的勝算(Odds)比Boys達到5+等級的勝算(Odds)高2倍等,與實際勝算比(Odds ratio)相比不會相差太多,代表弱符合比例賠率假設(Proportional odds assumption),可得到一個簡約的次序型分類模型。
	
\begin{table}[H]
	\footnotesize
    \centering
    \extrarowheight=5pt
    \caption{男孩和女孩英文成績水平的累積勝算(Odds)}\label{tab.3-4-2.1}
\setlength{\tabcolsep}{4mm}{
\begin{tabular}{llllll}
Boys                             & 3          & 4             & 5             & 6             & 7             \\\hline
Cumulative N boys                & 7177       & 6210          & 4838          & 2003          & 503           \\
Cumulative proportion            & 1.00       & 0.87          & 0.67          & 0.28          & 0.07          \\
Cumulative odds                  & -          & 6.42          & 2.07          & 0.39          & 0.08          \\
Cumulative logit                 &            & 1.86          & 0.73          & -0.95         & -2.59         \\
                                 &            &               &               &               &               \\
Girls                            & 3          & 4             & 5             & 6             & 7             \\\hline
Cumulative N boys                & 6987       & 6525          & 5621          & 2841          & 826           \\
Cumulative proportion            & 1.00       & 0.93          & 0.80          & 0.41          & 0.12          \\
Cumulative odds                  & -          & 14.12         & 4.11          & 0.69          & 0.13          \\
Cumulative logit                 & -          & 2.665         & 1.41          & -0.38         & -2.01         \\
                                 &            &               &               &               &               \\
\textbf{Odds Ratio (Girls/Boys)} & \textbf{-} & \textbf{2.20} & \textbf{1.99} & \textbf{1.77} & \textbf{1.78} \\
\textbf{Odds Ratio (Boys/Girls)} & \textbf{-} & \textbf{0.45} & \textbf{0.50} & \textbf{0.56} & \textbf{0.56}\\\hline
\end{tabular}
\begin{tablenotes}  
        \item[1.]\;\;\;\;\;\;\;\;\;\;\;\;此表取自SRME home 5.3 Key Assumptions of Ordinal Regression
\end{tablenotes}
}\end{table}
	
	
比例賠率假設的檢定問題如下:

\begin{equation} \nonumber \left\{\begin{array}{l} H_0: logit[P(Y \leq j)] = \beta_{0j} + \beta^{'}X \;\;\;\;\;;\;\;\; j = 1, ... , J-1\\   H_1: logit[P(Y \leq j)] = \beta_{0j} + \beta^{'}_jX \;\;\;\;;\;\;\; j = 1, ... , J-1  \end{array}\right.
\end{equation}

	本研究中我們以R package「ordinal」裡的nominal\underline{ }test檢驗該資料集是否符合比例賠率假設(Proportional odds assumption),該套件使用了概似比檢定(Likelihood ratio test)來做檢定。該檢定的檢定統計量為 -2($ln$(簡約模型) - $ln$(複雜模型))
	
	當解釋變數多、樣本量大的時候或者模型中包含連續的解釋變數,檢定結果幾乎會獲得非常小的P - value,因此應該更謹慎地做檢定,除了計算分數的檢定方法也可以用圖形判斷的方法做出更寬鬆的檢定。
	

	


	

\subsection{模擬方法}

  	本研究中所用的模擬資料生成方法是從Cumulative Logit Model抽樣,該模型之介紹見3.2.1章節,首先我們生成閾值$\theta_j$且閾值伴隨著j有上升的趨勢$\nonumber - \infty \equiv \theta_0 \leq  ... \leq  \theta_J \equiv \infty$,並初始化參數$\beta$固定為1,再來從常態分配生成$x_i$計算後可求得每一個觀測值$x_i$對應到的各類別機率,最後透過多項分配抽樣求得每個觀測值所對應的目標變數$r_i$。
  	

	模擬資料生成演算法如下表: 	 
\begin{algorithm}[H]   
	\caption{模擬資料生成方法, 由次序型分類模型的分配生成資料}   
	\label{alg.1}   
	\begin{algorithmic}[1]   	   	
		\State 初始化 $\beta$ = (1,1,...,1) $\in \mathbb{R}^f$ ;   	
		\State 從$N(0,1^2)$生成第一個閾值$\theta_1$且$\theta_1 \in \mathbb{R}$;   
		\State 閾值$\theta_j$滿足$\nonumber - \infty \equiv \theta_0 \leq  ... \leq  \theta_J \equiv \infty$;   		
		\Ensure   	 
		  模型參數$\beta$和${\theta_j}$生成次序值(反應變數)     
		\For{(i in 1:$N$)}{     	
			\State 從$N(0,1^2)$中生成$x_i$;
			\State 計算$x^{T}_i\beta$;     	     		
			\State 透過$F(x) = 1/(1+e^{-x})$,計算$P_r = F(\theta_j-x^{T}_i\beta) - F(\theta_{j-1}-x^{T}_i\beta)$;     		
			\State 針對每個$x_i$,透過多項分配來獲得對應的$r_i$     		
			\State $r_i \sim multinominal(N = 1, P_1, P_2,···, P_J )$; 	
		\EndFor
		} 		
 
         

	\end{algorithmic} 

\end{algorithm}


\section{詞嵌入 Word Embedding}
	詞嵌入是自然語言處理(NLP)中語言模型與表徵學習技術的統稱,文字訊息為非結構化資料,若要使用文字資訊,須將文字轉成能夠運算的數字或向量,將一段文字(文本)映射至一連續向量空間即為詞嵌入。
		
	詞向量是由大量的文字、單詞依照在句中出現次數等方式所訓練出來的,現今多使用TF-IDF的方法,以及Word2vec透過神經網絡針對現有的資料集訓練出的詞向量,或是使用網路上大公司如Google等使用超級電腦加上大量的文章數經由神經網絡所訓練出來的詞向量並結合詞袋模型(Bag of Words Model)。
	

\subsection{詞袋模型與TF-IDF}

	詞袋模型(Bag of Words Model)將對一篇文本中詞語的出現次數轉為數字,例如兩段話如下:
\begin{enumerate}[A.]
\setlength{\itemsep}{-10pt}
\item Jason and Chris are playing basketball. Jason has a basketball.\item Jason is playing computer.
\end{enumerate}

	經由上面兩段文字我們可以觀察到出現過的詞語有Jason / and / Chris / are / playing / basketball / has / a / is / computer,使用詞袋模型的方法能將兩句話表示成以下向量:
\begin{enumerate}[A.]
\setlength{\itemsep}{-10pt}
\item \text{[} 2, 1, 1, 1, 1, 2, 1, 1, 0, 0 \text{]}
\item \text{[} 1, 0, 0, 0, 1, 0, 0, 0, 1, 1 \text{]}
\end{enumerate}

	詞向量矩陣能表示出文本中的所有詞語,故稱為詞袋模型,傳統詞袋模型只使用出現次數,某種程度上較難判斷出每個詞語在文本中的重要性,另一種方式則是使用TF-IDF的方法調整詞袋模型的權重。
	
	TF-IDF由Jones(1972)一文中提出,包含了兩個部分,分別為詞頻(TF-Term Frequency)與逆向文件頻率(IDF-Inverse Document Frequency),為資料檢索中常使用到的工具,能夠量化每一個詞語在文本之中的重要程度。
	
	詞頻用來計算詞語在一篇文本中出現的頻率,愈常出現的詞語重要程度也更大,公式如式\ref{eq.3.2.1},其中$n_{ij}$代表第j篇文本第i個詞語出現的次數,分母項$n_{kj}$代表第j篇文本中總詞語數。
\begin{equation}\label{eq.3.2.1}
TF_{ij}=\frac{n_{ij}}{\sum_{K}^{}n_{kj}}
\end{equation}

	而詞頻會有常用字出現次數過多的問題,如「的」、「我」與「你」等等,這些詞語通常不是一篇文本中的主要詞語,但在計算詞頻時卻會將這些常用字高估,因此需要IDF來衡量詞語在文本中的普遍性。
	
	逆向文件頻率用來衡量一個詞語在所有文本中的普遍性,如式\ref{eq.3.2.2},其中D為總文本數,分母項為第i個詞語在所有文本中出現的文本次數,逆向文件頻率越小,代表一個詞語在越多文本中出現過,反之逆向文件頻率越大,代表一個詞語在越少文本中出現過,也就表示其不為常用字的可能性越大,所蘊含訊息也會更重要。
\begin{equation}\label{eq.3.2.2}
IDF_{i} = log\frac{D}{\left | \left \{ d_j:t_i\; \in d_j \right \} \right |}
\end{equation}
	
	TF-IDF計算方式則是將詞頻與逆向文件頻率相乘,得出式\ref{eq.3.2.3}。
\begin{equation}\label{eq.3.2.3}
TF-IDF_{ij}= TF_{ij} * IDF_j
\end{equation}

	然而,當有一個詞語在每一篇文本都出現過時,會使逆向文件頻率為0,計算TF-IDF時不論詞頻多大,結果都為0,考量到其可能會造成資訊損失,我們將逆向文件頻率公式加以平滑化,如式\ref{eq.3.2.4}。
	
\begin{equation}\label{eq.3.2.4}
IDF_{i}=log\frac{D+1}{\left | \left \{ d_j:t_i\; \in d_j \right \} \right |+1}+1
\end{equation}

	綜合上述,TF-IDF可以使我們衡量出每一個詞語在該文本中的特性,假設共有D篇文本,而所有文本共有V個詞語,可得出TF-IDF為一個D$\times$V的矩陣,以圖\ref{grap.3.3.1}表示TF-IDF法將文本中的文字轉為向量後的型態。
	
\begin{figure}[H]
    \centering
        \includegraphics[scale=0.7]{\imgdir TFIDF.png}
    \caption{TFIDF示意圖}
    \label{grap.3.3.1}
\end{figure}
	
	詞袋模型存在三個較為明顯的缺點:
\begin{enumerate}[A.]
\setlength{\itemsep}{-10pt}
\item 維度災難(curse of dimensionality)\\
	若將所有文本經由斷字後包含5000個詞語,使用詞袋模型將會導致詞向量維度高達5000維,對於電腦計算來說是相當高的維度且運算複雜,稱為維度災難。
\item 向量矩陣過於稀疏(sparse)\\
	若某篇文本與其他篇文本出現的共同詞語較少,例如其中一篇文本為醫學領域,其餘文本為經濟領域,在醫學領域中出現的專有名詞如冠狀病毒、粒線體等,將會導致其餘文本在這些詞與的向量表示為0,此為稀疏矩陣問題。
\item 無法表達前後文語意\\
	由於詞袋模型是將每個詞語獨立表示,因此無法表達出前後語句關係,解決方法為用N-Gram的方式將個別詞語以N個為單位一起表示。
\end{enumerate}
	
	我們可以只選擇較常出現的詞語,也就是選取較小的維度當作我們的詞向量以解決上述維度災難與稀疏矩陣的問題,後續也有學者提出各種方法用以得到更好的詞向量表達方式,以下我們將介紹Word2Vec的方法。
	
\subsection{CBOW (Continuous Bag Of Words)}
	Word2Vec為 Mikolov等(2013)提出透過神經網絡經訓練詞向量的方法,其中包含CBOW及Skip-gram兩種訓練方式。Word2Vec的想法是透過鄰近詞語來定義一個詞語的語意,經由淺層神經網絡去計算該詞語的詞向量,每個詞語都被用一個向量來表示,訓練出來的詞向量可以用餘弦去計算其相似度(Cosine Similarity),愈相關的詞語其相似度愈接近1,反之為0。
	
	Word2Vec的方法使一個詞語出現在不同的句子時,可以用很多不同的句子去定義該詞語的意思,例如我們可以透過下面兩句去訓練"Order"的詞向量。

\begin{enumerate}[A.]
\setlength{\itemsep}{-10pt}
\item I would like to make an order for a large pine table. 
\item I got an email saying that the order has been shipped.

\end{enumerate}	
	
	CBOW以文本為單位,給定上下文來預測中心詞,例如一句子「拜登成功當選2020年美國總統」,經由斷字後可得「拜登」、「成功』、「當選」、「2020年」、「美國」、「總統」,而CBOW算法會將「拜登」、「成功」、「2020年」、「美國」作為輸入,輸出為「當選」,而下一輪將會下移變成「成功」、「當選」、「美國」、「總統」作為輸入,輸出為「2020年」,輸入個數取決於參數Window Size,上述範例為 Window Size = 4,示意如圖\ref{grap.3.3.2}。
	
\begin{figure}[H]
    \centering
        \includegraphics[scale=0.7]{\imgdir CBOW.png}
    \caption{CBOW算法示意圖}
    \label{grap.3.3.2}
\end{figure}
	
	其中隱藏層的計算方式是針對語料庫中每一句Text的上下文c去計算中心詞w,透過最大化似然函數(likelihood)如式\ref{eq.3.3.2},即為最大化中心詞出現的機率乘積,以達到最大化模型參數$\theta$。而式\ref{eq.3.3.3}分子代表的是中心詞與上下詞與的內積,我們的目標就是讓上下文詞語與中心詞的向量盡可能接近,分母部分則是計算整個語料庫中所有上下文詞語分別與中心詞向量內積的總合,總得來看,式\ref{eq.3.3.3}會將此機率限縮至0到1之間,相當於一個softmax函數。
	
\begin{equation}\label{eq.3.3.2}
	arg\,\max\limits_{\theta}\prod_{w\in text}\left [ \prod_{c\in c(w))} p(w|c;\theta) \right ]
\end{equation}

\begin{equation}\label{eq.3.3.3}
	p(c|w;\theta) = \frac{e^{v_c\cdot{v_w} }}{\sum_{{c}'\in C} e^{v_{c}'\cdot{v_w}}}
\end{equation}
	
	而Word2vec透過神經網絡來訓練詞向量,透過輸入訓練樣本調整權重,讓整體預測的更準,也就是說 ,每個訓練樣本將會影響網絡中所有的權重,若訓練量一大,則會有運算速度上的問題,為解決這個問題,Mikolov(2013)提出負採樣(Negative Sampling),此方法的特點在於每次只修改一小部分的權重而不是全部,通常選擇K個(K通常為10到20)Negative word,來更新所對應的權重參數,這樣一來可使神經網絡在更新權重參數時計算速度更快,而加上負採樣後的隱藏層計算公式為式\ref{eq.3.3.4},其中$l$代表的是Logistic loss function。
	
\begin{equation}\label{eq.3.3.4}
arg\,\max\limits_{\theta}\sum_{w\in text}\left [ \sum_{c\in c(w))} l(s(w_t,w_c))+ \sum_{c\in N(t))} l(-s(w_t,w_c)) \right ]
\end{equation}

	而CBOW訓練出來的詞向量將會結合3.3.1中所提及的詞袋模型作為我們的文本向量(Sentence Encoding),如圖\ref{grap.3.3.3}為示,透過矩陣相乘後對每行向量取平均即為整個文本的輸入向量,其中CBOW所訓練出來的詞向量維度可以自行決定。
	
\begin{figure}[H]
    \centering
        \includegraphics[scale=0.7]{\imgdir CBOW2.png}
    \caption{Stenence Encoding 示意圖}
    \label{grap.3.3.3}
\end{figure}

\subsection{Skip-gram}
	Skip-gram與CBOW相反,以文本為單位,用中心詞來預測上下文,例如「拜登成功當選2020年美國總統」,經由斷字後可得「拜登」、「成功」、「當選」、「2020年」、「美國」、「總統」,而Skip-gram算法會將「當選」作為輸入,「拜登」、「成功」、「2020年」、「美國」作為輸出,而下一輪將會下移變成,「2020年」作為輸入,「成功」、「當選」、「美國」、「總統」作為輸出,,輸入個數取決於參數Window Size,上述範例為 Window Size = 4,示意圖如圖\ref{grap.3.4.1}。

\begin{figure}[H]
    \centering
        \includegraphics[scale=0.7]{\imgdir SKIPGRAM.png}
    \caption{Skip-gram算法示意圖}
    \label{grap.3.4.1}
\end{figure}

	Skip-gram算法於隱藏層計算如式\ref{eq.3.4.1},其餘大致上皆與CBOW相同,最後也是會結合詞袋模型去作文本向量(Sentence Encoding)。
	
\begin{equation}\label{eq.3.4.1}
	arg\,\max\limits_{\theta}\prod_{w\in text}\left [ \prod_{w\in c(w))} p(c|w;\theta) \right ]
\end{equation}

\subsection{Wiki Pretrain Word Embedding}

	介紹完Word2Vec的方法後,這邊將介紹一個新方法FastText,由Bojanowski等(2017)與Joulin等(2016)前後提出其概念,後由Facebook團隊開發,FastText與CBOW方式相似,差別在於目的性的不同,FastText目標是文字分類,詞向量為中間產出,CBOW目標是預測中心詞,以圖\ref{grap.3.4.1}為例,FastText輸出為一個標籤而非單詞,而FastText的輸入多考慮了子詞(Subword),這樣的特徵使得FastText在訓練時間上會花更久,而網路上也有比較FastText訓練出的詞向量與Word2Vec訓練出的詞向量,結果FastText所預測的更為準確。
	
\begin{figure}[H]
    \centering
        \includegraphics[scale=0.7]{\imgdir wiki.png}
    \caption{CBOW算法於Word2Vec與FastText比較}
    \label{grap.3.4.1}
\end{figure}

	在這邊我們將選用網路上的預訓練(Pretrain)詞向量,為Facebook使用維基百科上大量的語料庫結合高速運算電腦所訓練出來,與前兩者我們自己訓練的詞向量不同點在於,自己訓練的詞向量會更貼近於我們的資料集。

%\end{document}













%\notinput{preamble1}
%\usepackage{wallpaper}                                          % 使用浮水印
%\CenterWallPaper{0.6}{images/ntpu.eps}                           % 浮水印圖檔
%\begin{document}
%\fontsize{12}{22pt}\selectfont
\cleardoublepage
\thispagestyle{empty}
\setlength{\parindent}{2em}


\chapter{資料集介紹}

	此章節將介紹本次實驗所用的資料集與資料前處理,包含一般常見的數值型態和類別型態的資料共13筆,其目標變數經處理後皆為次序型,此外我們還使用文字資料,包含使用爬蟲程式抓取Yahoo電影上的中文文字資料以及Kaggle平台上Trip Advisor Hotel Reviews和Amazon Fine Food Reviews的英文文字資料。
	
\section{數值與類別型態資料}

	這13筆資料取自於UCI網站、Kaggle平台、網路上所提供的Ordinal Regression Benchmark Data與R\,Package上的資料,目標變數包含類別型與連續型,其中類別型的資料共有8筆且皆為次序型,其餘5筆資料我們參考Chu與Keerthi(2005)的研究中將目標變數為連續型的以等頻(Equal Frequency)的方式轉換為次序型,而因為有些資料集目標變數存在不平衡的問題,在這邊我們會分為沒有對目標變數做重抽樣與對目標變數做重抽樣兩部分做討論。
	

\subsection{全資料集總表}

	表\ref{tab.4.1.1.1}為13個資料集的基本性質,其中資料集7、8、9、10與13使用Chu與Keerthi(2005)等頻方法將目標變數由連續型轉為次序型,資料集最少有506筆最多為165474筆,變數個數則介於5到14。

\begin{table}[H]
	\footnotesize
    \centering
    \extrarowheight=5pt
    \caption{Data Statistics}\label{tab.4.1.1.1}
\setlength{\tabcolsep}{3mm}{
\begin{tabular}{ccccccc}
\hline
 ID& Data & Data Type & Dependent Varible & Task & Instances & Attributes \\ \hline
 1& CMC & Multivariate & Wife education &  Classification & 1473 & 10 \\ 
 2& CMC & Multivariate & Husband education &  Classification & 1473 & 10 \\ 
 3& CMC & Multivariate & Contraceptive &  Classification & 1473 & 10 \\ 
 4& soup & Multivariate & SURNESS &  Classification & 1847 & 7 \\ 
 5& affairs & Multivariate & rating &  Classification & 601 & 10 \\ 
 6& car  & Multivariate & acceptability &  Classification & 1728 & 7 \\ 
 7& e-shop  & Multivariate & order &  Regression & 165474 &11 \\ 
 8& estate  & Multivariate & CRIM &  Regression & 506 & 14 \\ 
 9& student.score  & Multivariate & score &  Regression & 1000 & 5 \\ 
 10& housing   & Multivariate & RH &  Regression & 20433 & 10 \\ 
 11& abalone  & Multivariate & Rings &  Classification & 4177 & 8 \\ 
 12& stock  & Multivariate & reward &  Classification & 950 & 10 \\ 
 13& fish.toxicity  & Multivariate & LC50 &  Regression & 908 & 7 \\ \hline
\end{tabular}}
\end{table}

	由於上述13筆資料集的目標變數大多皆為不平衡資料,我們將會在建立模型時對原始資料進行重抽樣(Resampling)。傳統的演算法在不平衡資料中具有較大的侷限性,例如若模型以全局性能做為學習過程指標,像是以正確率作為指標則會使預測結果大多傾向多數類別;樣本數過少的類別將會被視為離群值,從而降低該類別被預測率,但有可能離群值才是重點對象等等。
	
	常見的資料不平衡處理方法有:(1)重抽樣(Resampling)、(2)\,SMOTE與(3)\,Informed Undersampling。(1)的方法主要是將樣本數較多的類別依比例少抽,稱為欠採樣(Undersampling),較少的類別採抽後放回的方式依比例多抽,稱為過採樣(Oversampling),(2)與(3)的方法主要是在過採樣與欠採樣的過程加上演算法使其抽到有對資料本身做一點小變化,以產生不完全相同的資料,在這邊我們僅使用重抽樣。
	
	因此我們會針對每個資料集分為以下兩種形式作探討:
\begin{enumerate}[A.]
\setlength{\itemsep}{-10pt}
\item 未重抽樣資料建模,未重抽樣資料預測
\item 重抽樣資料建模,未重抽樣資料預測
\end{enumerate}

	我們會將所有資料集切分70\%與30\%做為訓練資料集與測試資料集,在切分訓練資料集時會依照目標變數各類別在資料集中佔的比例做等比例抽樣,意即若資料集之目標變數共有4類且各佔10\%、20\%、35\%與35\%且資料筆數為1000筆,那麼在切分後的訓練資料集目標變數各類別筆數分別為1000*10\%*70\%、1000*20\%*70\%、1000*35\%*70\%與1000*35\%*70\%,本研究中重抽樣會將各類別抽至與資料筆數最少的那類相同,為了預防有資料集資料筆數最少的類別筆數太少,若最少的類別筆數少於35筆時,會以倒數第二少的類別筆數為基準,將筆數較多的類別做欠採樣,筆數較少的類別做過採樣,表\ref{tab.4.1.1.2a}與\ref{tab.4.1.1.2b}為未做重抽樣與重抽樣下,各個資料集於訓練集、測試集與總資料的資料筆數。
	
	
\begin{table}[H]
	\footnotesize
    \centering
    \extrarowheight=3pt
    \caption{各資料集樣態 - A形式}\label{tab.4.1.1.2a}
\setlength{\tabcolsep}{1mm}{	
\begin{tabular}{lccccccccccccc}
             & \multicolumn{13}{c}{A    未重抽樣資料建模,未重抽樣資料預測}                                                                            \\ \cline{2-14} 
             & Data 1 & Data 2 & Data 3 & Data 4 & Data 5 & Data 6 & Data 7 & Data 8 & Data 9 & Data 10 & Data 11 & Data 12 & Data 13 \\ \hline
Training Set & 1029   & 1029   & 1030   & 1290   & 419  & 1208   & 115830 & 352    & 697    & 14301    & 2920     & 662      & 633     \\
Testing Set  & 444    & 444    & 443    & 557    & 182  & 520    & 49644  & 154    & 303    & 6132    & 1257     & 288       & 275     \\
Total Set    & 1473   & 1473   & 1473   & 1847   & 601  & 1728   & 165474 & 506    & 1000   & 20433    & 4177     & 950      & 908   \\\hline 
\end{tabular}
}\end{table}	


\begin{table}[H]
	\footnotesize
    \centering
    \extrarowheight=3pt
    \caption{各資料集樣態 - B形式}\label{tab.4.1.1.2b}
\setlength{\tabcolsep}{1mm}{	
\begin{tabular}{lccccccccccccc}
             & \multicolumn{13}{c}{B    重抽樣資料建模,未重抽樣資料預測}                                                                             \\ \cline{2-14} 
             & Data 1 & Data 2 & Data 3 & Data 4 & Data 5 & Data 6 & Data 7 & Data 8 & Data 9 & Data 10 & Data 11 & Data 12 & Data 13 \\ \hline
Training Set & 424    & 495    & 699    & 480    & 230   & 192    & 84090  & 350    & 675    & 14270    & 2184     & 550     & 630     \\
Testing Set  & 444    & 444    & 443    & 557    & 182  & 520    & 49644  & 154    & 303    & 6132    & 1257     & 288     & 275     \\
Total Set    & 868    & 939    & 1142   & 1037   & 412  & 712    & 133734 & 504    & 978    & 20402    & 3441     & 838     & 905   \\\hline 
\end{tabular}
}\end{table}


\subsection{各資料集簡介}
	
	資料集1、2與3取自UCI\;Data\;CMA,此數據集是1987年印尼避孕普及調查的其中一份資料集,這裡我們將目標變數分別設為wife\;ducation, husband\;education與Contraceptive,前兩者皆為妻子與丈夫的教育程度分級,由低至高分共為4個等級,而第三個是使用的避孕方法,包含不使用、短期使用與長期使用共三類。
	
	資料集4取自R 裡頭的ordinal\;package,由聯合利華研究公司提供,目標變數為SURNESS,依照測試者對於湯品的評分高低共分為六類。資料集6與7則是取自UCI\;Data,前者來自一個以汽車評估數據庫的簡單的分層決策模型,比如整體價格包含買入價與維修保養價格,技術特標包含舒適度與安全性等,並以整體可接受性作為我們的目標變數,共分為四等級;後者來自2008年前五個月,孕婦服裝在線商店的點擊流信息的相關資料,包含產品類別、價格等資訊,我們以ORDER點擊時間順序作為目標變數,因該變數為連續型,依照前面所敘述,我們以等頻方式切為五個等級。

	資料集5、8、9與10取自Kaggle平台的資料,資料集5為婚外情資料,目標變數是婚外情發生的頻率,共分為五個等級;資料集8為美國波士頓郊區房地產數據,以鎮上人均犯罪率做為目標變數,因該變數為連續型變數,我們以等頻方式切分低至高五個等級;資料集9為學生考試成績資料,透過家庭背景、考試準備等變數去了解對於學生成績的影響,考試成績共有數學、閱讀與寫作三者,我們將其取平均後以等頻方式切分為五個等級;資料集10則是1990年美國人口普查時所蒐集的加州房屋價格資料,我們以房屋價格中位數為目標變數以等頻方式切分為五個等級。
	
	資料集11、12取自網路上所提供的Ordinal Regression Benchmark Data,分別為預測鮑魚的環與股市趨勢,這兩組資料本身就已經先幫我們將連續型目標變數切分好等級,前者共切分為8個等級,後者切分為5個等級。資料集13為QSAR魚類毒性資料集,以各種毒物分子成份預測目標變數,該目標變數為連續型,我們以等頻方式切分為五個等級。
	
	表\ref{tab.4.1.1.3a}與\ref{tab.4.1.1.3b}為兩種形式下,各個資料集的目標變數於訓練集、測試集與總資料中各類別的分佈狀況。
	
\newpage
	
\begin{table}[H]
	\scriptsize
    \centering
    \extrarowheight=5pt
    \caption{各資料集目標變數樣態 - A形式}\label{tab.4.1.1.3a}
\setlength{\tabcolsep}{5mm}{
 \begin{tabular}{cccccccccc}
                         & \multicolumn{9}{c}{Class  of  Y}                                     \\ \cline{3-10} 
                         &       & 1     & 2     & 3     & 4     & 5     & 6    & 7    & 8    \\ \hline
\multirow{3}{*}{Data 1}  & Train & 106   & 233   & 287   & 403   &       &      &      &      \\
                         & Test  & 46    & 101   & 123   & 174   &       &      &      &      \\
                         & Total & 152   & 334   & 410   & 577   &       &      &      &      \\ \hline
\multirow{3}{*}{Data 2}  & Train & 30    & 124   & 246   & 629   &       &      &      &      \\
                         & Test  & 14    & 54    & 106   & 270   &       &      &      &      \\
                         & Total & 44    & 178   & 352   & 899   &       &      &      &      \\ \hline
\multirow{3}{*}{Data 3}  & Train & 440   & 233   & 357   &       &       &      &      &      \\
                         & Test  & 189   & 100   & 154   &       &       &      &      &      \\
                         & Total & 629   & 333   & 511   &       &       &      &      &      \\ \hline
\multirow{3}{*}{Data 4}  & Train & 159   & 182   & 80    & 68    & 193   & 608  &      &      \\
                         & Test  & 69    & 78    & 35    & 30    & 84    & 261  &      &      \\
                         & Total & 228   & 260   & 115   & 98    & 277   & 869  &      &      \\ \hline
\multirow{3}{*}{Data 5}  & Train & 11    & 46    & 65    & 135   & 162   &      & 	   & 	 \\
                         & Test  & 5     & 20    & 28    & 59  	 & 70    &      &      &   \\
                         & Total & 16    & 66    & 93    & 194   & 232   &      &      &  \\ \hline
\multirow{3}{*}{Data 6}  & Train & 847   & 268   & 48    & 45    &       &      &      &      \\
                         & Test  & 363   & 116   & 21    & 20    &       &      &      &      \\
                         & Total & 1210  & 384   & 69    & 65    &       &      &      &      \\ \hline
\multirow{3}{*}{Data 7}  & Train & 16818 & 24253 & 23422 & 26651 & 24686 &      &      &      \\
                         & Test  & 7208  & 10395 & 10039 & 11422 & 10580 &      &      &      \\
                         & Total & 24026 & 34648 & 33461 & 38073 & 35266 &      &      &      \\ \hline
\multirow{3}{*}{Data 8}  & Train & 70    & 70    & 70    & 70    & 72    &      &      &      \\
                         & Test  & 31    & 30    & 31    & 31    & 31    &      &      &      \\
                         & Total & 101   & 100   & 101   & 101   & 103   &      &      &      \\ \hline
\multirow{3}{*}{Data 9}  & Train & 143   & 135   & 138   & 143   & 138   &      &      &      \\
                         & Test  & 62    & 59    & 60    & 62    & 60    &      &      &      \\
                         & Total & 205   & 194   & 198   & 205   & 198   &      &      &      \\ \hline
\multirow{3}{*}{Data 10} & Train & 2860  & 2854  & 2866  & 2858  & 2863  &      &      &      \\
                         & Test  & 1226  & 1224  & 1229  & 1225  & 1228  &      &      &      \\
                         & Total & 4086  & 4078  & 4095  & 4083  & 4091  &      &      &      \\ \hline
\multirow{3}{*}{Data 11} & Train & 313   & 273   & 397   & 482   & 443   & 340  & 329  & 343  \\
                         & Test  & 135   & 118   & 171   & 207   & 191   & 147  & 141  & 147  \\
                         & Total & 448   & 391   & 568   & 689   & 634   & 487  & 470  & 490  \\ \hline
\multirow{3}{*}{Data 12} & Train & 110   & 158   & 190   & 144   & 60    &      &      &      \\
                         & Test  & 48    & 69    & 82    & 63    & 26    &      &      &      \\
                         & Total & 158   & 227   & 272   & 207   & 86    &      &      &      \\ \hline
\multirow{3}{*}{Data 13} & Train & 127   & 126   & 126   & 127   & 127   &      &      &      \\
                         & Test  & 55    & 55    & 55    & 55    & 55    &      &      &      \\
                         & Total & 182   & 181   & 181   & 182   & 182   &      &      &      \\ \hline
\end{tabular}
}\end{table}

\newpage

\begin{table}[H]
	\scriptsize
    \centering
    \extrarowheight=5pt
    \caption{各資料集目標變數樣態 - B形式}\label{tab.4.1.1.3b}
\setlength{\tabcolsep}{5mm}{
\begin{tabular}{cccccccccc}
                         & \multicolumn{9}{c}{Class of Y}                                     \\ \cline{3-10} 
                         &       & 1     & 2     & 3     & 4     & 5     & 6    & 7    & 8    \\ \hline
\multirow{3}{*}{Data 1}  & Train & 106   & 106   & 106   & 106   &       &      &      &      \\
                         & Test  & 46    & 101   & 123   & 174   &       &      &      &      \\
                         & Total & 152   & 207   & 229   & 280   &       &      &      &      \\ \hline
\multirow{3}{*}{Data 2}  & Train & 124   & 124   & 123   & 124   &       &      &      &      \\
                         & Test  & 14    & 54    & 106   & 270   &       &      &      &      \\
                         & Total & 138   & 178   & 229   & 394   &       &      &      &      \\ \hline
\multirow{3}{*}{Data 3}  & Train & 233   & 233   & 233   &       &       &      &      &      \\
                         & Test  & 189   & 100   & 154   &       &       &      &      &      \\
                         & Total & 422   & 333   & 387   &       &       &      &      &      \\ \hline
\multirow{3}{*}{Data 4}  & Train & 80    & 80    & 80    & 80    & 80    & 80   &      &      \\
                         & Test  & 69    & 78    & 35    & 30    & 84    & 261  &      &      \\
                         & Total & 149   & 158   & 115   & 110   & 164   & 341  &      &      \\ \hline
\multirow{3}{*}{Data 5}  & Train & 46    & 46    & 46    & 46    & 46    &      &      &   \\
                         & Test  & 5     & 20    & 28    & 59    & 70    &      &      &   \\
                         & Total & 51    & 66    & 74    & 105   & 116   &      &      &  \\ \hline
\multirow{3}{*}{Data 6}  & Train & 48    & 48    & 48    & 48    &       &      &      &      \\
                         & Test  & 363   & 116   & 21    & 20    &       &      &      &      \\
                         & Total & 411   & 164   & 69    & 68    &       &      &      &      \\ \hline
\multirow{3}{*}{Data 7}  & Train & 16818 & 16818 & 16818 & 16818 & 16818 &      &      &      \\
                         & Test  & 7208  & 10395 & 10039 & 11422 & 10580 &      &      &      \\
                         & Total & 24026 & 27213 & 26857 & 28240 & 27398 &      &      &      \\ \hline
\multirow{3}{*}{Data 8}  & Train & 70    & 70    & 70    & 70    & 70    &      &      &      \\
                         & Test  & 31    & 30    & 31    & 31    & 31    &      &      &      \\
                         & Total & 101   & 100   & 101   & 101   & 101   &      &      &      \\ \hline
\multirow{3}{*}{Data 9}  & Train & 135   & 135   & 135   & 135   & 135   &      &      &      \\
                         & Test  & 62    & 59    & 60    & 62    & 60    &      &      &      \\
                         & Total & 197   & 194   & 195   & 197   & 195   &      &      &      \\ \hline
\multirow{3}{*}{Data 10} & Train & 2854  & 2854  & 2854  & 2854  & 2854  &      &      &      \\
                         & Test  & 1226  & 1224  & 1229  & 1225  & 1228  &      &      &      \\
                         & Total & 4080  & 4078  & 4083  & 4079  & 4082  &      &      &      \\ \hline
\multirow{3}{*}{Data 11} & Train & 273   & 273   & 273   & 273   & 273   &  273 &  273 &  273 \\
                         & Test  & 135   & 118   & 171   & 207   & 191   &  147 &  141 &  147 \\
                         & Total & 408   & 391   & 444   & 480   & 464   &  420 &  414 &  420 \\ \hline
\multirow{3}{*}{Data 12} & Train & 110   & 110   & 110   & 110   & 110   &      &      &      \\
                         & Test  & 48    & 69    & 82    & 63    & 26    &      &      &      \\
                         & Total & 158   & 179   & 192   & 173   & 136   &      &      &      \\ \hline
\multirow{3}{*}{Data 13} & Train & 126   & 126   & 126   & 126   & 126   &      &      &      \\
                         & Test  & 55    & 55    & 55    & 55    & 55    &      &      &      \\
                         & Total & 181   & 181   & 181   & 181   & 181   &      &      &      \\ \hline
\end{tabular}
}\end{table}
 

	
\section{文字型態資料}
	
	除了一般數值與類別型態的資料,我們也將實驗文字資料於目標變數為次序型和名目型的比較何者較佳,下面我們將會介紹這兩筆文字形態資料,並執行資料前處理,文字種類包含中文與英文,Yahoo電影評論、Kaggle平台上提供的Trip Advisor Hotel Reviews。


\subsection{中文文字資料 - Yahoo電影評論}

	我們使用Python進行網路爬蟲,鎖定Yahoo電影網站上2020年上映的國片,時間為該片上映後至2021/02/28,擷取每部電影網友所留的評論與該評論所對應的評分,評分介於1至5分,合計共36部電影,評分愈高代表愈滿意。
	
	為了排除評論數少量的電影,使電影更具有代表性,我們挑選每部電影評論數10則以上作為有效電影,其餘則捨棄,詳細電影名稱如表\ref{tab.4.2.1.1},合計共24部有效電影,而評論方面,我們扣除沒有文字的評論後的有效評論共2885則。
	
\begin{table}[H]
	\small
    \centering
    \extrarowheight=5pt
    \caption{有效與無效電影名稱}\label{tab.4.2.1.1}
\setlength{\tabcolsep}{3mm}{
\begin{tabular}{c|c|c}
     & 有效電影                                                                                                                                                                                                                      & 無效電影                                                                                                                    \\ \hline
電影名稱 & \begin{tabular}[c]{@{}l@{}}同學麥娜絲, 驚夢49天, 逃出立法院, \\ 打噴嚏, 馗降:粽邪2,破處, \\ 怪胎, 初心, 刻在你心底的名字, \\ 可不可以,你也剛好喜歡,\\ 我消失的情人節, 海霧, 你的情歌, \\ 女鬼橋, 腿,老娘就要這麼活, \\ 孤味, 親愛的房客, 無聲 , \\ 哈囉少女,杏林醫院, 戀愛iNG, \\ 親愛的殺手, 十二夜2:回到第零天\end{tabular} & \begin{tabular}[c]{@{}l@{}}迷走廣州, 媽!我阿榮啦, 野雀之師, \\ 菠蘿蜜, 千年一問, 狂飆一夢, \\ 蚵豐村, 惡之畫, 戀愛好好說, \\ 逆者, 阿紫, 我的兒子是死刑犯\end{tabular}
\end{tabular}
}\end{table}

\begin{figure}[H]
    \centering
        \includegraphics[scale=0.8]{\imgdir yahoo_rating.png}
    \caption{篩選有效評論前後評分的分佈對照}
    \label{grap.4.2.1}
\end{figure}

	針對文本資料前處理的部分,首先,將所有評論刪去表情符號(Emoji)、標點符號和換行的符號,再來我們對每一則評論使用Jieba斷詞,最後我們刪除繁體中文常見停用字的部分,例如「的」和「了」等詞語對整篇文本來說太常出現且沒有太大的解釋意義,這些停用字若不刪除可能會造成後續做詞向量時效果變差,以上為文本資料前處理的部分,效果如表\ref{tab.4.2.1.2}。
	
\begin{table}[H]
	\small
    \centering
    \extrarowheight=5pt
    \caption{中文文本處理前後對照}\label{tab.4.2.1.2}
\setlength{\tabcolsep}{3mm}{
\begin{tabular}{c|c}
         & 第2881則評論                                                                \\ \hline
文本處理前 & \begin{tabular}[c]{@{}l@{}}真的很棒!影片在談論的議題。\\ 許多人的努力,非常值得一看。\end{tabular} \\ \hline
文本處理後 & {[}'真的', '很棒', '影片', '談論', '議題', '許多人', '努力', '值得一看'{]}                
\end{tabular}
}\end{table}

\subsection{英文文字資料 - Trip Advisor Hotel Reviews}

	此資料集為Kaggle平台上所提供的Trip Advisor Hotel Reviews,於Tripadvisor網站上抓取的20491則旅店公開評論所組合而成的資料集,資料期間不詳,評分介於1至5分,評論中皆沒有出現缺失值,故有效評論共20491則。
	
\begin{figure}[H]
    \centering
        \includegraphics[scale=1.1]{\imgdir tripadvisor_rating.png}
    \caption{Trip Advisor評分分佈}
    \label{grap4.2.2}
\end{figure}
	
	針對文本資料前處理的部分,首先使用Python軟體中「nltk」套件裡的「tokenize」將所有評論做斷詞,接下來將英文停用字,例如「so」與「too」等單字,最後我們將所有評論的表情符號(Emoji)、標點符號等非英文與數字的字刪掉。效果如表\ref{tab.4.2.2.1}

\begin{table}[H]
	\small
    \centering
    \extrarowheight=5pt
    \caption{英文文本處理前後對照(Trip Advisor)}\label{tab.4.2.2.1}
\setlength{\tabcolsep}{3mm}{
\begin{tabular}{c|c}
      & 第480則評論                                                                                                                                                            \\ \hline
文本處理前 & beware beware leave vehicle, took advantage park ride unfortunately vehicle broken,                                                                                \\ \hline
文本處理後 & \begin{tabular}[c]{@{}l@{}}{[}'beware', 'beware', 'leave', 'vehicle', 'took', 'advantage', \\     \,\,\,\,\,\,\,    'park', 'ride', 'unfortunately', 'vehicle', 'broken'{]}\end{tabular}

\end{tabular}
}\end{table}



\newpage
%\end{document}











%\input{preamble1}
%\usepackage{wallpaper}                                          % 使用浮水印
%\CenterWallPaper{0.6}{images/ntpu.eps}                           % 浮水印圖檔
%\begin{document}
\fontsize{12}{22pt}\selectfont
\cleardoublepage
\thispagestyle{empty}
\setlength{\parindent}{2em}
\chapter{實例分析與模擬研究}
	本章節會列出13筆數值與類別型態資料於第3節介紹之五種分類模型之下的預測結果,以及利用統計模擬的方法產生符合比例賠率假設(Proportional odds assumption)的模擬資料於五種模型的預測結果,並會比較不平衡資料重抽樣與否對於結果的差異,最後則是會將中英文評論資料搭配不同詞嵌入方法處理後,比較在次序型分類模型與名目型分類模型的預測結果。
	
	本研究Cumulative Logit Model使用Christensen(2019)提出的ordinal:: clm;Continuation-Ratio Logit Model與Adjacent-Category Logit Model使用Yee(2020)提出的VGAM:: vglm;樸素貝葉斯使用Meyer等(2021)提出的e1071:: naiveBayes;邏輯斯模型使用Venables(2002)提出的nnet:: multinom;比例賠率假設檢定使用Christensen(2019)提出的ordinal:: nominal\underline{ }test。

\section{實例分析}
	首先我們使用每個資料集中的70\%做為訓練集建立五種模型,並將其餘30\%做為測試集利用衡量指標檢視模型預測能力,總共有13組資料集,每個資料集我們將重複執行50次並將結果紀錄以檢視其平均與標準差,如4.1.1中所介紹,將分為未重抽樣建模與重抽樣建模來看,表5.3和表5.4為各資料集於A.「未重抽樣資料建模」與B.「重抽樣資料建模」兩種方法建模並預測測試集之概觀結果,比較模型預測能力,而表5.1(a)-(m)為13組資料集於A方法的詳細結果,表5.2(a)-(m)為13組資料集於B方法的詳細結果,其中各衡量指標(MSE、Accuracy、Macro-F1)表現最佳以顏色底標示。綜合三指標結果得出各組資料對應之最佳模型,比方從表5.1(a)可看出邏輯斯模型表現最佳,以此類推,將結果整理於表5.3。
	
	從表5.3(a)與表5.4(a)中可以發現到名目型分類模型與次序型分類模型的表現是差不多的,前者7組資料及表現較優,後者在6組資料及表現較優,表現優劣勢根據MSE、Accuracy以及Macro-F1三種衡量指標的綜合表現,而從表5.3(b)與表5.4(b)中可以觀察到各模型中以多元邏輯斯模型為最佳,這也讓我們疑惑,為何次序型資料在次序型分類模型中沒有表現的較好,參考到Agresti(2003)一書中所提到Cumulative Logit Model必需建立在比例賠率假設(Proportional odds assumption)之下,關於此假設詳細說明在本文3.3.1,符合比例賠率假設之下的Cumulative Logit Model優點是易於解釋與總結,且每個預測變量只需要一個參數,下面我們將檢定這13組資料是否符合比例賠率假設,結果顯示一般而言,當資料符合該假設時,次序型分類模型的預測能力優於名目行分類模型,反之則較差,檢定結果詳列於表5.5。表5.5為對13組資料集使用概似比檢定的結果,關於此假設與檢定在章節3.3.1節。與表5.3比較可以發現,在次序型分類模型表現較好的資料集2、4、5、7、9與13中不違反此假設的變數比例較高,其餘資料集違反此假設的變數比例較高的則是在名目型分類模型表現較佳。
	

\newpage

% BIGTABLE A
\begin{table}[H]
\caption{各資料集於A方法未重抽樣建模,未重抽樣預測於五種分類模型預測結果}
\label{table.5.1.A}


\begin{subtable}[H]{.5\textwidth}
	\footnotesize
    \centering
    \extrarowheight=5pt
\setlength{\tabcolsep}{2mm}{
\begin{tabular}{cccc}
\hline
Model  & MSE          & Accuracy     & Macro-F1     \\ \hline
Clm    & \cellcolor[HTML]{FFCE93}0.676(0.045) & 0.562(0.019) & 0.506(0.025) \\ 
Naive  & 0.85(0.231)  & 0.556(0.035) & 0.496(0.035) \\ 
Logis  & 0.681(0.048) & \cellcolor[HTML]{FFCE93}0.574(0.018) & \cellcolor[HTML]{FFCE93}0.525(0.022) \\ 
cratio & 0.684(0.047) & 0.568(0.019) & 0.504(0.024) \\ 
acat   & 0.703(0.042) & 0.554(0.018) & 0.491(0.024) \\ \hline
\end{tabular}

\caption{Dataset1於A方法五種分類模型預測結果}\label{tab.5.1.1a}
\vspace{0.5cm}
}\end{subtable}
\hfill
\begin{subtable}[H]{.6\linewidth}
	\footnotesize
    \centering
    \extrarowheight=5pt
\setlength{\tabcolsep}{2mm}{
\begin{tabular}{cccc}
\hline
Model  & MSE          & Accuracy     & Macro-F1     \\ \hline
Clm    & \cellcolor[HTML]{FFCE93}0.522(0.034) & 0.647(0.019) & \cellcolor[HTML]{FFCE93}0.476(0.047) \\ 
Naive  & 0.814(0.479) & 0.617(0.045) & 0.405(0.032) \\ 
Logis  & 0.545(0.037) & 0.646(0.016) & 0.469(0.043) \\ 
cratio & 0.53(0.035)  & 0.645(0.018) & 0.47(0.041)  \\ 
acat   & 0.55(0.033)  & \cellcolor[HTML]{FFCE93}0.647(0.017) & 0.463(0.049) \\ \hline
\end{tabular}

    \caption{Dataset2於A方法五種分類模型預測結果}\label{tab.5.1.2a}
    \vspace{0.5cm}
}\end{subtable}
\hfill

\begin{subtable}[H]{.5\linewidth}
	\footnotesize
    \centering
    \extrarowheight=5pt
\setlength{\tabcolsep}{2mm}{
\begin{tabular}{cccc}
\hline
Model  & MSE          & Accuracy     & Macro-F1     \\ \hline
Clm    & 1.39(0.069)  & 0.483(0.017) & 0.538(0.02)  \\ 
Naive  & \cellcolor[HTML]{FFCE93}1.135(0.087) & 0.469(0.024) & 0.468(0.024) \\ 
Logis  & 1.244(0.067) & \cellcolor[HTML]{FFCE93}0.512(0.021) & 0.487(0.023) \\ 
cratio & 1.381(0.068) & 0.485(0.017) & \cellcolor[HTML]{FFCE93}0.542(0.02)  \\ 
acat   & 1.388(0.066) & 0.484(0.016) & 0.539(0.019) \\ \hline 

\end{tabular}

\caption{Dataset3於A方法五種分類模型預測結果}\label{tab.5.1.3a}
\vspace{0.5cm}
}\end{subtable}
\hfill
\begin{subtable}[H]{.6\linewidth}
	\footnotesize
    \centering
    \extrarowheight=5pt
\setlength{\tabcolsep}{2mm}{
\begin{tabular}{cccc}
\hline
Model  & MSE          & Accuracy     & Macro-F1     \\ \hline
Clm    & 6.271(0.008) & \cellcolor[HTML]{FFCE93}0.469(0)     & \cellcolor[HTML]{FFCE93}0.638(0)     \\ 
Naive  & \cellcolor[HTML]{FFCE93}6.23(0.096)  & 0.469(0.005) & 0.364(0.083) \\ 
Logis  & 6.249(0.066) & 0.468(0.004) & 0.365(0.1)   \\ 
cratio & 6.277(0.036) & 0.468(0.002) & 0.621(0.07)  \\ 
acat   & 6.269(0)     & \cellcolor[HTML]{FFCE93}0.469(0)     & \cellcolor[HTML]{FFCE93}0.638(0)     \\ \hline
\end{tabular}

    \caption{Dataset4於A方法五種分類模型預測結果}\label{tab.5.1.4a}
    \vspace{0.5cm}
}\end{subtable}

\begin{subtable}[H]{.5\textwidth}
	\footnotesize
    \centering
    \extrarowheight=5pt
\setlength{\tabcolsep}{2mm}{
\begin{tabular}{cccc}
\hline
Model  & MSE          & Accuracy     & Macro-F1     \\ \hline
Clm    & \cellcolor[HTML]{FFCE93}0.676(0.045) & 0.562(0.019) & 0.506(0.025) \\ 
Naive  & 0.85(0.231)  & 0.556(0.035) & 0.496(0.035) \\ 
Logis  & 0.681(0.048) & \cellcolor[HTML]{FFCE93}0.574(0.018) & \cellcolor[HTML]{FFCE93}0.525(0.022) \\ 
cratio & 0.684(0.047) & 0.568(0.019) & 0.504(0.024) \\ 
acat   & 0.703(0.042) & 0.554(0.018) & 0.491(0.024) \\ \hline
\end{tabular}

\caption{Dataset5於A方法五種分類模型預測結果}\label{tab.5.1.5a}
\vspace{0.5cm}
}\end{subtable}
\hfill
\begin{subtable}[H]{.6\linewidth}
	\footnotesize
    \centering
    \extrarowheight=5pt
\setlength{\tabcolsep}{2mm}{
\begin{tabular}{cccc}
\hline
Model  & MSE          & Accuracy     & Macro-F1     \\ \hline
Clm    & 0.082(0.009) & 0.922(0.008) & 0.801(0.028) \\ 
Naive  & 0.208(0.021) & 0.851(0.013) & 0.647(0.04)  \\ 
Logis  & \cellcolor[HTML]{FFCE93}0.08(0.012)  & \cellcolor[HTML]{FFCE93}0.934(0.009) & \cellcolor[HTML]{FFCE93}0.88(0.028)  \\ 
cratio & 0.082(0.01)  & 0.921(0.008) & 0.8(0.03)    \\ 
acat   & 0.084(0.009) & 0.921(0.008) & 0.791(0.027) \\ \hline
\end{tabular}

    \caption{Dataset6於A方法五種分類模型預測結果}\label{tab.5.1.6a}
    \vspace{0.5cm}
}\end{subtable}
\hfill
\begin{subtable}[H]{.5\linewidth}
	\footnotesize
    \centering
    \extrarowheight=5pt
\setlength{\tabcolsep}{2mm}{
\begin{tabular}{cccc}
\hline
Model  & MSE          & Accuracy     & Macro-F1     \\ \hline
Clm    & \cellcolor[HTML]{FFCE93}2.477(0.012) & 0.278(0.002) & 0.257(0.002) \\ 
Naive  & 2.914(0.043) & 0.272(0.002) & 0.232(0.002) \\ 
Logis  & 2.68(0.022)  & \cellcolor[HTML]{FFCE93}0.283(0.001) & 0.232(0.002) \\ 
cratio & 2.548(0.014) & 0.279(0.002) & \cellcolor[HTML]{FFCE93}0.34(0.002)  \\ 
acat   & 2.501(0.013) & 0.278(0.001) & 0.256(0.001) \\ \hline
\end{tabular}

\caption{Dataset7於A方法五種分類模型預測結果}\label{tab.5.1.7a}
\vspace{0.5cm}
}\end{subtable}
\hfill
\begin{subtable}[H]{.6\linewidth}
	\footnotesize
    \centering
    \extrarowheight=5pt
\setlength{\tabcolsep}{2mm}{
\begin{tabular}{cccc}
\hline
Model  & MSE          & Accuracy     & Macro-F1     \\ \hline
Clm    & \cellcolor[HTML]{FFCE93}0.543(0.057) & 0.568(0.03)  & 0.56(0.029)  \\ 
Naive  & 0.7(0.08)    & 0.565(0.028) & 0.546(0.031) \\ 
Logis  & 0.547(0.092) & \cellcolor[HTML]{FFCE93}0.644(0.029) & \cellcolor[HTML]{FFCE93}0.64(0.03)   \\ 
cratio & 0.555(0.053) & 0.558(0.03)  & 0.552(0.029) \\ 
acat   & 0.563(0.054) & 0.567(0.027) & 0.556(0.027) \\ \hline
\end{tabular}

    \caption{Dataset8於A方法五種分類模型預測結果}\label{tab.5.1.8a}
    \vspace{0.5cm}
}\end{subtable}
\begin{subtable}[H]{.5\textwidth}
	\footnotesize
    \centering
    \extrarowheight=5pt
\setlength{\tabcolsep}{2mm}{
\begin{tabular}{cccc}
\hline
Model  & MSE          & Accuracy     & Macro-F1     \\ \hline
Clm    & 3.218(0.171) & 0.288(0.019) & 0.256(0.018) \\ 
Naive  & \cellcolor[HTML]{FFCE93}3.111(0.223) & 0.272(0.019) & 0.259(0.019) \\ 
Logis  & 3.094(0.21)  & 0.273(0.022) & 0.26(0.021)  \\ 
cratio & 3.391(0.18)  & 0.291(0.022) & \cellcolor[HTML]{FFCE93}0.292(0.033) \\ 
acat   & 3.49(0.213)  & \cellcolor[HTML]{FFCE93}0.292(0.023) & 0.27(0.035)  \\ \hline
\end{tabular}

\caption{Dataset9於A方法五種分類模型預測結果}\label{tab.5.1.9a}
\vspace{0.5cm}
}\end{subtable}
\hfill
\begin{subtable}[H]{.6\linewidth}
	\footnotesize
    \centering
    \extrarowheight=5pt
\setlength{\tabcolsep}{2mm}{
\begin{tabular}{cccc}
\hline
Model  & MSE          & Accuracy     & Macro-F1     \\ \hline
Clm    & \cellcolor[HTML]{FFCE93}0.771(0.017) & 0.525(0.005) & 0.528(0.005) \\ 
Naive  & 1.33(0.037)  & 0.473(0.006) & 0.447(0.009) \\ 
Logis  & 0.792(0.017) & \cellcolor[HTML]{FFCE93}0.558(0.005) & \cellcolor[HTML]{FFCE93}0.554(0.005) \\ 
cratio & 0.788(0.015) & 0.516(0.006) & 0.517(0.005) \\ 
acat   & 0.788(0.017) & 0.523(0.005) & 0.522(0.005) \\ \hline
\end{tabular}

    \caption{Dataset10於A方法五種分類模型預測結果}\label{tab.5.1.10a}
    \vspace{0.5cm}
}\end{subtable}
\end{table}

\begin{table}[]
\ContinuedFloat
\begin{subtable}[H]{.5\linewidth}
	\footnotesize
    \centering
    \extrarowheight=5pt
\setlength{\tabcolsep}{2mm}{
\begin{tabular}{cccc}
\hline
Model  & MSE          & Accuracy    & Macro-F1     \\ \hline
Clm    & 2.548(0.074) & 0.331(0.01) & 0.345(0.023) \\ 
Naive  & 4.149(0.252) & 0.31(0.01)  & 0.317(0.02)  \\ 
Logis  & \cellcolor[HTML]{FFCE93}2.513(0.102) & \cellcolor[HTML]{FFCE93}0.367(0.01) & \cellcolor[HTML]{FFCE93}0.352(0.01)  \\ 
cratio & 2.686(0.09)  & 0.311(0.01) & 0.352(0.02)  \\ 
acat   & 2.602(0.086) & 0.324(0.01) & 0.318(0.011) \\ \hline
\end{tabular}

\caption{Dataset11於A方法五種分類模型預測結果}\label{tab.5.1.11a}
\vspace{0.5cm}
}\end{subtable}
\hfill
\begin{subtable}[H]{.6\linewidth}
	\footnotesize
    \centering
    \extrarowheight=5pt
\setlength{\tabcolsep}{2mm}{
\begin{tabular}{cccc}
\hline
Model  & MSE          & Accuracy     & Macro-F1     \\ \hline
Clm    & 0.326(0.026) & 0.697(0.022) & 0.727(0.022) \\ 
Naive  & 0.314(0.037) & 0.747(0.018) & 0.753(0.017) \\ 
Logis  & \cellcolor[HTML]{FFCE93}0.194(0.035) & \cellcolor[HTML]{FFCE93}0.835(0.023) & \cellcolor[HTML]{FFCE93}0.846(0.023) \\ 
cratio & 0.325(0.027) & 0.698(0.023) & 0.728(0.022) \\ 
acat   & 0.331(0.027) & 0.692(0.023) & 0.722(0.022) \\ \hline
\end{tabular}

    \caption{Dataset12於A方法五種分類模型預測結果}\label{tab.5.1.12a}
    \vspace{0.5cm}
}\end{subtable}


\begin{subtable}[H]{.5\linewidth}
	\footnotesize
    \centering
    \extrarowheight=5pt
\setlength{\tabcolsep}{2mm}{
\begin{tabular}{cccc}
\hline
Model  & MSE          & Accuracy     & Macro-F1     \\ \hline
Clm    & \cellcolor[HTML]{FFCE93}0.981(0.12)  & \cellcolor[HTML]{FFCE93}0.511(0.026) & \cellcolor[HTML]{FFCE93}0.513(0.025) \\ 
Naive  & 1.21(0.103)  & 0.473(0.03)  & 0.455(0.032) \\ 
Logis  & 1.073(0.11)  & 0.482(0.029) & 0.474(0.028) \\ 
cratio & 0.994(0.113) & 0.5(0.028)   & 0.5(0.026)   \\ 
acat   & 1.018(0.116) & 0.507(0.024) & 0.503(0.022) \\ \hline
\end{tabular}

    \caption{Dataset13於A方法五種分類模型預測結果}\label{tab.5.1.13a}
    \vspace{0.5cm}
}\end{subtable}
\end{table}


% BIGTABLE B
\begin{table}[H]
\caption{各資料集於B方法未重抽樣建模,未重抽樣預測於五種分類模型預測結果}
\label{table.5.1.B}

\begin{subtable}[H]{.5\textwidth}
	\footnotesize
    \centering
    \extrarowheight=5pt
\setlength{\tabcolsep}{2mm}{
\begin{tabular}{cccc}
\hline
Model  & MSE          & Accuracy     & Macro-F1     \\ \hline
Clm    & 0.745(0.049) & 0.511(0.02)  & 0.482(0.019) \\ 
Naive  & 1.276(0.644) & 0.49(0.077)  & 0.434(0.056) \\ 
Logis  & 0.842(0.069) & \cellcolor[HTML]{FFCE93}0.534(0.024) & \cellcolor[HTML]{FFCE93}0.492(0.022) \\ 
cratio & \cellcolor[HTML]{FFCE93}0.73(0.047)  & 0.528(0.021) & 0.49(0.019)  \\ 
acat   & 0.778(0.048) & 0.515(0.02)  & 0.479(0.019) \\ \hline
\end{tabular}

\caption{Dataset1於B方法五種分類模型預測結果}\label{tab.5.1.1b}
\vspace{0.5cm}
}\end{subtable}
\hfill
\begin{subtable}[H]{.6\linewidth}
	\footnotesize
    \centering
    \extrarowheight=5pt
\setlength{\tabcolsep}{2mm}{
\begin{tabular}{cccc}
\hline
Model  & MSE          & Accuracy     & Macro-F1     \\ \hline
Clm    & 0.738(0.071) & 0.605(0.024) & 0.453(0.024) \\ 
Naive  & 1.049(0.592) & 0.585(0.059) & 0.415(0.043) \\ 
Logis  & 0.865(0.108) & 0.6(0.022)   & 0.429(0.026) \\ 
cratio & \cellcolor[HTML]{FFCE93}0.688(0.055) & \cellcolor[HTML]{FFCE93}0.621(0.021) & \cellcolor[HTML]{FFCE93}0.456(0.023) \\ 
acat   & 0.795(0.077) & 0.604(0.023) & 0.439(0.024) \\ \hline
\end{tabular}

    \caption{Dataset2於B方法五種分類模型預測結果}\label{tab.5.1.2b}
    \vspace{0.5cm}
}\end{subtable}
\hfill
\begin{subtable}[H]{.5\linewidth}
	\footnotesize
    \centering
    \extrarowheight=5pt
\setlength{\tabcolsep}{2mm}{
\begin{tabular}{cccc}
\hline
Model  & MSE          & Accuracy     & Macro-F1     \\ \hline
Clm    & 1.143(0.062) & 0.465(0.028) & 0.438(0.028) \\ 
Naive  & 1.095(0.085) & 0.456(0.022) & 0.455(0.023) \\ 
Logis  & \cellcolor[HTML]{FFCE93}1.086(0.072) & \cellcolor[HTML]{FFCE93}0.506(0.022) & \cellcolor[HTML]{FFCE93}0.502(0.022) \\ 
cratio & 1.15(0.065)  & 0.462(0.025) & 0.437(0.024) \\ 
acat   & 1.192(0.062) & 0.472(0.026) & 0.432(0.027) \\ \hline
\end{tabular}

\caption{Dataset3於B方法五種分類模型預測結果}\label{tab.5.1.3b}
\vspace{0.5cm}
}\end{subtable}
\hfill
\begin{subtable}[H]{.6\linewidth}
	\footnotesize
    \centering
    \extrarowheight=5pt
\setlength{\tabcolsep}{2mm}{
\begin{tabular}{cccc}
\hline
Model  & MSE          & Accuracy     & Macro-F1     \\ \hline
Clm    & 7.197(0.447) & 0.304(0.02)  & 0.199(0.023) \\ 
Naive  & \cellcolor[HTML]{FFCE93}6.103(0.401) & 0.253(0.025) & 0.19(0.017)  \\ 
Logis  & 6.237(0.424) & 0.25(0.027)  & 0.192(0.019) \\ 
cratio & 7.153(0.459) & 0.321(0.02)  & \cellcolor[HTML]{FFCE93}0.232(0.041) \\ 
acat   & 7.668(0.471) & \cellcolor[HTML]{FFCE93}0.325(0.018) & 0.195(0.025) \\ \hline
\end{tabular}

    \caption{Dataset4於B方法五種分類模型預測結果}\label{tab.5.1.4b}
    \vspace{0.5cm}
}\end{subtable}

\hfill
\end{table}

\begin{table}[H]
\ContinuedFloat


\begin{subtable}[H]{.5\textwidth}
	\footnotesize
    \centering
    \extrarowheight=5pt
\setlength{\tabcolsep}{2mm}{
\begin{tabular}{cccc}
\hline
Model  & MSE          & Accuracy     & Macro-F1     \\ \hline
Clm    & 2.652(0.298) & 0.302(0.034) & 0.231(0.026) \\ 
Naive  & 2.723(0.397) & 0.321(0.041) & \cellcolor[HTML]{FFCE93}0.253(0.033) \\ 
Logis  & 2.825(0.323) & 0.297(0.038) & 0.244(0.036) \\ 
cratio & \cellcolor[HTML]{FFCE93}2.354(0.281) & \cellcolor[HTML]{FFCE93}0.332(0.027) & 0.246(0.025) \\ 
acat   & 2.735(0.332) & 0.312(0.029) & 0.231(0.024) \\ \hline
\end{tabular}

\caption{Dataset5於B方法五種分類模型預測結果}\label{tab.5.1.5b}
\vspace{0.5cm}
}\end{subtable}
\hfill
\begin{subtable}[H]{.6\linewidth}
	\footnotesize
    \centering
    \extrarowheight=5pt
\setlength{\tabcolsep}{2mm}{
\begin{tabular}{cccc}
\hline
Model  & MSE          & Accuracy     & Macro-F1     \\ \hline
Clm    & \cellcolor[HTML]{FFCE93}0.097(0.016) & \cellcolor[HTML]{FFCE93}0.914(0.012) & \cellcolor[HTML]{FFCE93}0.821(0.026) \\ 
Naive  & 0.223(0.025) & 0.825(0.018) & 0.747(0.03)  \\ 
Logis  & 0.419(0.183) & 0.85(0.033)  & 0.756(0.049) \\ 
cratio & 0.099(0.022) & 0.913(0.018) & 0.821(0.03)  \\ 
acat   & 0.121(0.026) & 0.892(0.022) & 0.806(0.031) \\ \hline
\end{tabular}

    \caption{Dataset6於B方法五種分類模型預測結果}\label{tab.5.1.6b}
    \vspace{0.5cm}
}\end{subtable}

\begin{subtable}[H]{.5\linewidth}
	\footnotesize
    \centering
    \extrarowheight=5pt
\setlength{\tabcolsep}{2mm}{
\begin{tabular}{cccc}
\hline
Model  & MSE          & Accuracy     & Macro-F1     \\ \hline
Clm    & \cellcolor[HTML]{FFCE93}3.34(0.02)   & 0.258(0.002) & \cellcolor[HTML]{FFCE93}0.245(0.002) \\ 
Naive  & 4.139(0.035) & \cellcolor[HTML]{FFCE93}0.245(0.002) & 0.203(0.002) \\ 
Logis  & 3.777(0.043) & 0.257(0.005) & 0.225(0.01)  \\ 
cratio & 3.438(0.019) & 0.262(0.002) & 0.24(0.001)  \\ 
acat   & 3.639(0.02)  & 0.255(0.002) & 0.228(0.002) \\ \hline
\end{tabular}

\caption{Dataset7於B方法五種分類模型預測結果}\label{tab.5.1.7b}
\vspace{0.5cm}
}\end{subtable}
\hfill
\begin{subtable}[H]{.6\linewidth}
	\footnotesize
    \centering
    \extrarowheight=5pt
\setlength{\tabcolsep}{2mm}{
\begin{tabular}{cccc}
\hline
Model  & MSE          & Accuracy     & Macro-F1     \\ \hline
Clm    & \cellcolor[HTML]{FFCE93}0.562(0.063) & 0.567(0.031) & 0.561(0.032) \\ 
Naive  & 0.73(0.116)  & 0.567(0.034) & 0.543(0.04)  \\ 
Logis  & 0.562(0.103) & \cellcolor[HTML]{FFCE93}0.642(0.032) & \cellcolor[HTML]{FFCE93}0.637(0.033) \\ 
cratio & 0.565(0.053) & 0.563(0.03)  & 0.558(0.032) \\ 
acat   & 0.568(0.054) & 0.565(0.03)  & 0.557(0.031) \\ \hline
\end{tabular}

    \caption{Dataset8於B方法五種分類模型預測結果}\label{tab.5.1.8b}
    \vspace{0.5cm}
}\end{subtable}
\begin{subtable}[H]{.5\textwidth}
	\footnotesize
    \centering
    \extrarowheight=5pt
\setlength{\tabcolsep}{2mm}{
\begin{tabular}{cccc}
\hline
Model  & MSE          & Accuracy     & Macro-F1     \\ \hline
Clm    & 3.141(0.241) & 0.293(0.025) & \cellcolor[HTML]{FFCE93}0.269(0.028) \\ 
Naive  & \cellcolor[HTML]{FFCE93}3.04(0.21)   & 0.265(0.024) & 0.255(0.023) \\ 
Logis  & 3.086(0.235) & 0.266(0.023) & 0.256(0.022) \\ 
cratio & 3.22(0.223)  & 0.288(0.027) & 0.259(0.027) \\ 
acat   & 3.403(0.225) & \cellcolor[HTML]{FFCE93}0.29(0.021)  & 0.251(0.021) \\ \hline
\end{tabular}

\caption{Dataset9於B方法五種分類模型預測結果}\label{tab.5.1.9b}
\vspace{0.5cm}
}\end{subtable}
\hfill
\begin{subtable}[H]{.6\linewidth}
	\footnotesize
    \centering
    \extrarowheight=5pt
\setlength{\tabcolsep}{2mm}{
\begin{tabular}{cccc}
\hline
Model  & MSE          & Accuracy     & Macro-F1     \\ \hline
Clm    & \cellcolor[HTML]{FFCE93}0.77(0.014)  & 0.525(0.006) & 0.528(0.006) \\ 
Naive  & 1.339(0.045) & 0.473(0.007) & 0.448(0.011) \\ 
Logis  & 0.794(0.014) & \cellcolor[HTML]{FFCE93}0.557(0.004) & \cellcolor[HTML]{FFCE93}0.554(0.004) \\ 
cratio & 0.787(0.013) & 0.516(0.006) & 0.518(0.006) \\ 
acat   & 0.787(0.015) & 0.523(0.006) & 0.522(0.006) \\ \hline
\end{tabular}

    \caption{Dataset10於B方法五種分類模型預測結果}\label{tab.5.1.10b}
    \vspace{0.5cm}
}\end{subtable}

\begin{subtable}[H]{.5\linewidth}
	\footnotesize
    \centering
    \extrarowheight=5pt
\setlength{\tabcolsep}{2mm}{
\begin{tabular}{cccc}
\hline
Model  & MSE          & Accuracy     & Macro-F1     \\ \hline
Clm    & 2.66(0.089)  & 0.333(0.011) & 0.341(0.011) \\ 
Naive  & 4.362(0.319) & 0.302(0.01)  & 0.314(0.021) \\ 
Logis  & \cellcolor[HTML]{FFCE93}2.715(0.136) & \cellcolor[HTML]{FFCE93}0.356(0.011) & \cellcolor[HTML]{FFCE93}0.351(0.012) \\ 
cratio & 2.83(0.089)  & 0.316(0.011) & 0.323(0.011) \\ 
acat   & 2.661(0.095) & 0.329(0.012) & 0.33(0.013)  \\ \hline
\end{tabular}

\caption{Dataset11於B方法五種分類模型預測結果}\label{tab.5.1.11b}
\vspace{0.5cm}
}\end{subtable}
\hfill
\begin{subtable}[H]{.6\linewidth}
	\footnotesize
    \centering
    \extrarowheight=5pt
\setlength{\tabcolsep}{2mm}{
\begin{tabular}{cccc}
\hline
Model  & MSE          & Accuracy     & Macro-F1     \\ \hline
Clm    & \cellcolor[HTML]{FFCE93}0.329(0.028) & 0.693(0.025) & 0.724(0.022) \\ 
Naive  & 0.331(0.043) & 0.744(0.02)  & 0.747(0.02)  \\ 
Logis  & 0.223(0.045) & \cellcolor[HTML]{FFCE93}0.819(0.022) & \cellcolor[HTML]{FFCE93}0.831(0.02)  \\ 
cratio & 0.33(0.028)  & 0.693(0.024) & 0.724(0.022) \\ 
acat   & 0.33(0.027)  & 0.692(0.024) & 0.724(0.022) \\ \hline
\end{tabular}

    \caption{Dataset12於B方法五種分類模型預測結果}\label{tab.5.1.12b}
    \vspace{0.5cm}
}\end{subtable}

\begin{subtable}[H]{.5\linewidth}
	\footnotesize
    \centering
    \extrarowheight=5pt
\setlength{\tabcolsep}{2mm}{
\begin{tabular}{cccc}
\hline
Model  & MSE          & Accuracy     & Macro-F1     \\ \hline
Clm    & \cellcolor[HTML]{FFCE93}1.003(0.108) & \cellcolor[HTML]{FFCE93}0.504(0.033) & \cellcolor[HTML]{FFCE93}0.505(0.032) \\ 
Naive  & 1.252(0.124) & 0.465(0.028) & 0.451(0.031) \\ 
Logis  & 1.141(0.098) & 0.475(0.026) & 0.467(0.026) \\ 
cratio & 1.017(0.107) & 0.49(0.031)  & 0.49(0.031)  \\ 
acat   & 1.042(0.104) & 0.5(0.031)   & 0.496(0.031) \\ \hline
\end{tabular}

    \caption{Dataset13於B方法五種分類模型預測結果}\label{tab.5.1.13b}
    \vspace{0.5cm}
}\end{subtable}
\end{table}

% table A
\begin{table}[H]
\centering
\caption{各資料集於A方法預測狀況}
\label{tab.5.1.A1}

\begin{subtable}[H]{.5\linewidth}
	\footnotesize
    \extrarowheight=5pt
	\setlength{\tabcolsep}{5mm}{
	\begin{tabular}{ccc}
	\hline
        & 名目型分類模型                   & 次序型分類模型              \\ \hline
Dataset & 1, 3, 6, 8, 10, 11, 12 & 2, 4, 5, 7, 9, 13\\
	\hline
\end{tabular}
    \caption{各資料集於分類模型預測成效}\label{tab.5.1.a1}
}\end{subtable}

\hfill
\vspace{0.5cm}

\begin{subtable}[H]{.78\linewidth}
	\footnotesize
    \extrarowheight=5pt
	\setlength{\tabcolsep}{5mm}{
	\begin{tabular}{cccccc}
	\hline
        & Clm      & Naive & Logistic               & cratio & acat \\ \hline
Dataset & 2, 5, 13 &       & 1, 3, 6, 8, 10, 11, 12 & 7, 9   & 4   \\ \hline
\end{tabular}
    \caption{各資料集於五種分類模型預測成效}\label{tab.5.1.a2}
}\end{subtable}

\end{table}

% table B
\begin{table}[H]
\centering
\caption{各資料集於B方法預測狀況}
\label{tab.5.1.B1}

\begin{subtable}[H]{.5\linewidth}
	\footnotesize
    \extrarowheight=5pt
	\setlength{\tabcolsep}{5mm}{
\begin{tabular}{ccc}
\hline
        & 名目型分類模型                   & 次序型分類模型              \\ \hline
Dataset & 1, 6, 8, 10, 11, 12 & 2, 3, 4, 5, 7, 9, 13 \\ \hline
\end{tabular}
    \caption{各資料集於分類模型預測成效}\label{tab.5.1.b1}
}\end{subtable}

\hfill
\vspace{0.5cm}


\begin{subtable}[H]{.78\linewidth}
	\footnotesize
    \extrarowheight=5pt
	\setlength{\tabcolsep}{5mm}{
\begin{tabular}{cccccc}
\hline
        & Clm         & Naive & Logistic            & cratio  & acat \\ \hline
Dataset & 6, 7, 9, 13 &       & 1, 3, 8, 10, 11, 12 & 2, 4, 5 &     \\ \hline
\end{tabular}
    \caption{各資料集於五種分類模型預測成效}\label{tab.5.1.b2}
}\end{subtable}
\end{table}


\begin{table}[H]
	\small
    \centering
    \extrarowheight=5pt
    \caption{13組資料集概似比檢定結果}\label{tab.5.1.2}
\setlength{\tabcolsep}{2mm}{
\begin{tabular}{cccccccc}
\hline
Dataset                                                                & 1                                                                   & 2                                                                          & 3                          & 4                                                        & 5                                                                               & 6                                                                                       & 7                        \\ \hline
\begin{tabular}[c]{@{}c@{}}Variables\\ (不含Y)\end{tabular}                & 9                                                                  & 9                                                                         & 9 & 6                                                        & 9                                                                              & 6                                                                                       & 10                       \\ \hline
\begin{tabular}[c]{@{}c@{}}不拒絕比例賠率假設 \\ 之變數 \end{tabular} & \begin{tabular}[c]{@{}c@{}}V4, V5, V6,\\  V9, V10\end{tabular}      & \begin{tabular}[c]{@{}c@{}}V2, V3, V4,\\ V5, V6, V7,\\ V8, V9\end{tabular} & V6                         & \begin{tabular}[c]{@{}c@{}}V2, V3,\\ V6, V7\end{tabular} & \begin{tabular}[c]{@{}c@{}}V2, V3, V4,\\ V5, V6, V7,\\ V8, V9, V10\end{tabular} & V4                                                                                      & V8, V10                  \\ \hline
\multicolumn{1}{l}{}                                                   & \multicolumn{1}{l}{}                                                & \multicolumn{1}{l}{}                                                       & \multicolumn{1}{l}{}       & \multicolumn{1}{l}{}                                     & \multicolumn{1}{l}{}                                                            & \multicolumn{1}{l}{}                                                                    & \multicolumn{1}{l}{}     \\ \hline
Dataset                                                                & 8                                                                   & 9                                                                          & 10                         & 11                                                       & 12                                                                              & 13                                                                                      &                          \\ \hline
\begin{tabular}[c]{@{}c@{}}Variables\\ (不含Y)\end{tabular}                & 13                                                                  & 4                                                                          & 9 & 6                                                        & 9                                                                              & 6                                                                                       &  \\ \hline
\begin{tabular}[c]{@{}c@{}}不拒絕比例賠率假設 \\ 之變數 \end{tabular} & \begin{tabular}[c]{@{}c@{}}V2, V3, V4,\\  V6, V13, V14 \\ \end{tabular} & \begin{tabular}[c]{@{}c@{}}V2, V3,\\ V4,V5\end{tabular}                    &                            &                                                          &                                                                                 & \begin{tabular}[c]{@{}c@{}}V2, V3, V4\\ V5, V6, V7\end{tabular} &                          \\ \hline
\end{tabular}
}\end{table}


	根據上述結論讓我們不禁好奇,雖然過去傳統統計模型強調「解釋能力」(配適能力),但若符合前提假設也就是模型配適的更好之下,是否也保證有較佳的預測能力,下章節將使用統計模擬驗證此論證。


\section{統計模擬}

	本章節我們將利用統計模擬的方法來驗證,當資料符合模型前提假設之下,除了有良好的解釋能力以外,同時也會有較佳的預測能力,透過由次序型分類模型的分配生成資料的統計模擬的方法,我們分別產生1到10個維度的資料並將反應變數固定在1到5的區間,每個資料集的資料筆數皆為50000筆,而各模擬資料皆為不平衡資料如表\ref{tab.5.2.1},詳細模擬資料生成方法在本文3.3.1。

	使用前面所提到的概似比檢定來檢定上述模擬資料生成的方式是否符合比例賠率假設,檢定結果顯示該方法所生成的10組資料的各個變數皆符合比例賠率假設。

\begin{table}[H]
	\small
    \centering
    \extrarowheight=5pt
    \caption{模擬資料目標變數樣態}\label{tab.5.2.1}
\setlength{\tabcolsep}{5mm}{
\begin{tabular}{cccccc}
\hline
Y      & 1     & 2     & 3     & 4     & 5     \\ \hline
Data1  & 3259  & 3030  & 1501  & 16495 & 25715 \\
Data2  & 29030 & 9141  & 11768 & 18    & 43    \\
Data3  & 40795 & 2002  & 2134  & 276   & 4793  \\
Data4  & 31196 & 2570  & 4396  & 3933  & 7905  \\
Data5  & 18721 & 4495  & 17750 & 5417  & 3617  \\
Data6  & 30149 & 3101  & 8100  & 4207  & 4443  \\
Data7  & 31122 & 12727 & 1981  & 3269  & 901   \\
Data8  & 23641 & 3966  & 2221  & 6466  & 13700 \\
Data9  & 26105 & 6745  & 4948  & 7246  & 4956  \\
Data10 & 24746 & 2671  & 4165  & 4265  & 14153 \\ \hline
\end{tabular}
}\end{table}


	由表\ref{tab.5.2.11}中可以觀察到當資料符合模型前提假設,除了模型具有解釋能力以外,預測能力也較佳,其中由表\ref{tab.5.2.a2}來看又以Cumulative Logit Model表現最佳。然而在表\ref{tab.5.2.3}中可以看到B方法使用重抽樣資料去建模時次序型分類模型表現較佳的資料集會減少。
	
\newpage

\begin{table}[H]
\centering
\caption{模擬資料於A方法預測狀況}
\label{tab.5.2.11}

\begin{subtable}[H]{.5\linewidth}
	\footnotesize
    \extrarowheight=5pt
	\setlength{\tabcolsep}{5mm}{
	\begin{tabular}{ccc}
	\hline
        & 名目型分類模型                   & 次序型分類模型              \\ \hline
Dataset & 4, 10 & 1, 2, 3, 5, 6, 7, 8, 9, 10\\
	\hline
\end{tabular}
    \caption{模擬資料於分類模型預測成效}\label{tab.5.2.a1}
}\end{subtable}

\hfill
\vspace{0.5cm}

\begin{subtable}[H]{.78\linewidth}
	\footnotesize
    \extrarowheight=5pt
	\setlength{\tabcolsep}{5mm}{
	\begin{tabular}{cccccc}
	\hline
        & Clm      & Naive & Logistic               & cratio & acat \\ \hline
Dataset & 1, 2, 3, 5, 6, 7, 8, 9 &       & 4, 10 & 7, 8   & 3, 10   \\ \hline
\end{tabular}
    \caption{模擬資料於五種分類模型預測成效}\label{tab.5.2.a2}
}\end{subtable}

\end{table}

\begin{table}[H]
\centering
\caption{模擬資料於B方法預測狀況}
\label{tab.5.2.3}

\begin{subtable}[H]{.5\linewidth}
	\footnotesize
    \extrarowheight=5pt
	\setlength{\tabcolsep}{5mm}{
	\begin{tabular}{ccc}
	\hline
        & 名目型分類模型                   & 次序型分類模型              \\ \hline
Dataset & 1, 4, 8, 9, 10 & 2, 3, 4, 5, 6, 7\\
	\hline
\end{tabular}
    \caption{模擬資料於分類模型預測成效}\label{tab.5.2.b1}
}\end{subtable}

\hfill
\vspace{0.5cm}

\begin{subtable}[H]{.78\linewidth}
	\footnotesize
    \extrarowheight=5pt
	\setlength{\tabcolsep}{5mm}{
	\begin{tabular}{cccccc}
	\hline
        & Clm      & Naive & Logistic               & cratio & acat \\ \hline
Dataset & 2, 5, 6, 7, 8, 9 & 1,       & 8, 10 &  & 3, 4  \\ \hline
\end{tabular}
    \caption{模擬資料於五種分類模型預測成效}\label{tab.5.2.b2}
}\end{subtable}

\end{table}

\newpage

% BIGTABLE C
\begin{table}[H]
\caption{模擬資料於A方法未重抽樣建模,未重抽樣預測於五種分類模型預測結果}
\label{table.5.2.A}


\begin{subtable}[H]{.5\textwidth}
	\footnotesize
    \centering
    \extrarowheight=5pt
\setlength{\tabcolsep}{2mm}{
\begin{tabular}{cccc}
\hline
Model  & MSE          & Accuracy     & Macro-F1     \\ \hline
Clm    & \cellcolor[HTML]{FFCE93}1.427(0.01)  & \cellcolor[HTML]{FFCE93}0.563(0.003) & \cellcolor[HTML]{FFCE93}0.413(0.004) \\
Naive  & 1.428(0.01)  & 0.562(0.003) & 0.407(0.004) \\
Logis  & 1.443(0.009) & 0.562(0.003) & 0.392(0.004) \\
Cratio & 1.435(0.009) & 0.562(0.003) & 0.403(0.003) \\
acat   & 1.656(0.014) & 0.553(0.002) & 0.403(0.004) \\ \hline
\end{tabular}

\caption{模擬資料1維於A方法五種模型預測結果}\label{tab.5.2.1a}
\vspace{0.5cm}
}\end{subtable}
\hfill
\begin{subtable}[H]{.6\linewidth}
	\footnotesize
    \centering
    \extrarowheight=5pt
\setlength{\tabcolsep}{2mm}{
\begin{tabular}{cccc}
\hline
Model  & MSE                                 & Accuracy                             & Macro-F1                             \\ \hline
Clm    & \cellcolor[HTML]{FFCE93}0.829(0.01) & \cellcolor[HTML]{FFCE93}0.656(0.002) & \cellcolor[HTML]{FFCE93}0.665(0.004) \\
Naive  & 0.834(0.008)                        & 0.655(0.002)                         & 0.656(0.003)                         \\
Logis  & 0.83(0.01)                          & \cellcolor[HTML]{FFCE93}0.656(0.002) & \cellcolor[HTML]{FFCE93}0.665(0.003) \\
Cratio & 0.83(0.009)                         & \cellcolor[HTML]{FFCE93}0.656(0.002) & 0.663(0.003)                         \\
acat   & 0.83(0.01)                          & \cellcolor[HTML]{FFCE93}0.656(0.002) & \cellcolor[HTML]{FFCE93}0.665(0.004) \\ \hline
\end{tabular}

    \caption{模擬資料2維於A方法五種模型預測結果}\label{tab.5.2.2a}
    \vspace{0.5cm}
}\end{subtable}
\hfill

\begin{subtable}[H]{.5\linewidth}
	\footnotesize
    \centering
    \extrarowheight=5pt
\setlength{\tabcolsep}{2mm}{
\begin{tabular}{cccc}
\hline
Model  & MSE                                  & Accuracy                             & Macro-F1                             \\ \hline
Clm    & \cellcolor[HTML]{FFCE93}1.543(0.018) & \cellcolor[HTML]{FFCE93}0.833(0.001) & 0.663(0.005)                         \\
Naive  & 1.586(0.014)                         & 0.83(0.001)                          & 0.604(0.005)                         \\
Logis  & 1.547(0.019)                         & \cellcolor[HTML]{FFCE93}0.833(0.001) & \cellcolor[HTML]{FFCE93}0.666(0.005) \\
Cratio & 1.551(0.019)                         & 0.832(0.001)                         & 0.647(0.006)                         \\
acat   & \cellcolor[HTML]{FFCE93}1.543(0.019) & \cellcolor[HTML]{FFCE93}0.833(0.001) & 0.664(0.005)                         \\ \hline
\end{tabular}

\caption{模擬資料3維於A方法五種模型預測結果}\label{tab.5.2.3a}
\vspace{0.5cm}
}\end{subtable}
\hfill
\begin{subtable}[H]{.6\linewidth}
	\footnotesize
    \centering
    \extrarowheight=5pt
\setlength{\tabcolsep}{2mm}{
\begin{tabular}{cccc}
\hline
Model  & MSE                                 & Accuracy                             & Macro-F1                             \\ \hline
Clm    & 2.285(0.027)                        & \cellcolor[HTML]{FFCE93}0.697(0.002) & 0.705(0.003)                         \\
Naive  & 2.432(0.025)                        & 0.692(0.001)                         & 0.688(0.004)                         \\
Logis  & \cellcolor[HTML]{FFCE93}2.28(0.029) & \cellcolor[HTML]{FFCE93}0.697(0.002) & \cellcolor[HTML]{FFCE93}0.706(0.003) \\
Cratio & 2.3(0.028)                          & \cellcolor[HTML]{FFCE93}0.697(0.002) & 0.703(0.004)                         \\
acat   & 2.283(0.028)                        & \cellcolor[HTML]{FFCE93}0.697(0.002) & 0.705(0.003)                         \\ \hline
\end{tabular}

    \caption{模擬資料4維於A方法五種模型預測結果}\label{tab.5.2.4a}
    \vspace{0.5cm}
}\end{subtable}

\begin{subtable}[H]{.5\textwidth}
	\footnotesize
    \centering
    \extrarowheight=5pt
\setlength{\tabcolsep}{2mm}{
\begin{tabular}{cccc}
\hline
Model  & MSE                                  & Accuracy                             & Macro-F1                             \\ \hline
Clm    & \cellcolor[HTML]{FFCE93}1.048(0.011) & \cellcolor[HTML]{FFCE93}0.599(0.003) & 0.509(0.005)                         \\
Naive  & 1.126(0.011)                         & 0.585(0.002)                         & 0.532(0.041)                         \\
Logis  & 1.07(0.012)                          & 0.598(0.003)                         & 0.487(0.006)                         \\
Cratio & 1.08(0.011)                          & 0.597(0.003)                         & \cellcolor[HTML]{FFCE93}0.619(0.004) \\
acat   & 1.086(0.013)                         & 0.597(0.003)                         & \cellcolor[HTML]{FFCE93}0.617(0.005) \\ \hline
\end{tabular}

\caption{模擬資料5維於A方法五種模型預測結果}\label{tab.5.2.5a}
\vspace{0.5cm}
}\end{subtable}
\hfill
\begin{subtable}[H]{.6\linewidth}
	\footnotesize
    \centering
    \extrarowheight=5pt
\setlength{\tabcolsep}{2mm}{
\begin{tabular}{cccc}
\hline
Model  & MSE                                  & Accuracy                             & Macro-F1                             \\ \hline
Clm    & \cellcolor[HTML]{FFCE93}1.235(0.015) & \cellcolor[HTML]{FFCE93}0.674(0.002) & 0.447(0.003)                         \\
Naive  & 1.77(0.019)                          & 0.651(0.002)                         & 0.489(0.005)                         \\
Logis  & 1.279(0.016)                         & 0.673(0.002)                         & \cellcolor[HTML]{FFCE93}0.575(0.021) \\
Cratio & 1.385(0.016)                         & 0.672(0.002)                         & 0.558(0.004)                         \\
acat   & 1.297(0.017)                         & 0.673(0.002)                         & \cellcolor[HTML]{FFCE93}0.575(0.004) \\ \hline
\end{tabular}

    \caption{模擬資料6維於A方法五種模型預測結果}\label{tab.5.2.6a}
    \vspace{0.5cm}
}\end{subtable}
\hfill
\begin{subtable}[H]{.5\linewidth}
	\footnotesize
    \centering
    \extrarowheight=5pt
\setlength{\tabcolsep}{2mm}{
\begin{tabular}{cccc}
\hline
Model  & MSE                                  & Accuracy                            & Macro-F1                             \\ \hline
Clm    & \cellcolor[HTML]{FFCE93}0.518(0.008) & \cellcolor[HTML]{FFCE93}0.73(0.003) & 0.547(0.009)                         \\
Naive  & 0.717(0.006)                         & 0.693(0.002)                        & 0.329(0.005)                         \\
Logis  & 0.524(0.008)                         & 0.729(0.003)                        & 0.533(0.01)                          \\
Cratio & 0.523(0.008)                         & \cellcolor[HTML]{FFCE93}0.73(0.003) & \cellcolor[HTML]{FFCE93}0.562(0.009) \\
acat   & 0.537(0.008)                         & 0.726(0.003)                        & 0.538(0.01)                          \\ \hline
\end{tabular}

\caption{模擬資料7維於A方法五種模型預測結果}\label{tab.5.2.7a}
\vspace{0.5cm}
}\end{subtable}
\hfill
\begin{subtable}[H]{.6\linewidth}
	\footnotesize
    \centering
    \extrarowheight=5pt
\setlength{\tabcolsep}{2mm}{
\begin{tabular}{cccc}
\hline
Model  & MSE                                  & Accuracy                             & Macro-F1                             \\ \hline
Clm    & \cellcolor[HTML]{FFCE93}2.061(0.024) & \cellcolor[HTML]{FFCE93}0.678(0.002) & 0.549(0.003)                         \\
Naive  & 2.296(0.025)                         & 0.672(0.001)                         & 0.762(0.002)                         \\
Logis  & 2.142(0.028)                         & 0.677(0.002)                         & 0.608(0.125)                         \\
Cratio & 2.16(0.026)                          & \cellcolor[HTML]{FFCE93}0.677(0.001) & \cellcolor[HTML]{FFCE93}0.765(0.002) \\
acat   & 2.141(0.028)                         & 0.677(0.002)                         & \cellcolor[HTML]{FFCE93}0.764(0.002) \\ \hline
\end{tabular}

    \caption{模擬資料8維於A方法五種模型預測結果}\label{tab.5.2.8a}
    \vspace{0.5cm}
}\end{subtable}
\begin{subtable}[H]{.5\textwidth}
	\footnotesize
    \centering
    \extrarowheight=5pt
\setlength{\tabcolsep}{2mm}{
\begin{tabular}{cccc}
\hline
Model  & MSE                                  & Accuracy                             & Macro-F1                             \\ \hline
Clm    & \cellcolor[HTML]{FFCE93}1.095(0.014) & \cellcolor[HTML]{FFCE93}0.647(0.002) & 0.525(0.004)                         \\
Naive  & 1.6(0.019)                           & 0.612(0.002)                         & 0.564(0.006)                         \\
Logis  & 1.162(0.017)                         & 0.645(0.002)                         & 0.5(0.016)                           \\
Cratio & 1.212(0.019)                         & 0.643(0.002)                         & \cellcolor[HTML]{FFCE93}0.639(0.004) \\
acat   & 1.205(0.018)                         & 0.644(0.002)                         & \cellcolor[HTML]{FFCE93}0.639(0.004) \\ \hline
\end{tabular}

\caption{模擬資料9維於A方法五種模型預測結果}\label{tab.5.2.9a}
\vspace{0.5cm}
}\end{subtable}
\hfill
\begin{subtable}[H]{.6\linewidth}
	\footnotesize
    \centering
    \extrarowheight=5pt
\setlength{\tabcolsep}{2mm}{
\begin{tabular}{cccc}
\hline
Model  & MSE                                  & Accuracy                            & Macro-F1                             \\ \hline
Clm    & 1.97(0.03)                           & 0.709(0.002)                        & \cellcolor[HTML]{FFCE93}0.788(0.002) \\
Naive  & 2.113(0.029)                         & 0.703(0.002)                        & 0.784(0.002)                         \\
Logis  & \cellcolor[HTML]{FFCE93}1.964(0.03)  & \cellcolor[HTML]{FFCE93}0.71(0.002) & \cellcolor[HTML]{FFCE93}0.788(0.002) \\
Cratio & 1.981(0.03)                          & 0.709(0.002)                        & \cellcolor[HTML]{FFCE93}0.788(0.002) \\
acat   & \cellcolor[HTML]{FFCE93}1.963(0.029) & \cellcolor[HTML]{FFCE93}0.71(0.002) & \cellcolor[HTML]{FFCE93}0.788(0.002) \\ \hline
\end{tabular}

    \caption{模擬資料10維於A方法五種模型預測結果}\label{tab.5.2.10a}
    \vspace{0.5cm}
}\end{subtable}
\end{table}


% BIGTABLE D
\begin{table}[H]
\caption{模擬資料於B方法重抽樣建模,未重抽樣預測於五種模型預測結果}
\label{table.5.2.B}


\begin{subtable}[H]{.5\textwidth}
	\footnotesize
    \centering
    \extrarowheight=5pt
\setlength{\tabcolsep}{2mm}{
\begin{tabular}{cccc}
\hline
Model  & MSE                                  & Accuracy                             & Macro-F1                             \\ \hline
Clm    & 2.437(0.036)                         & 0.428(0.005)                         & 0.277(0.003)                         \\
Naive  & \cellcolor[HTML]{FFCE93}2.225(0.153) & 0.444(0.01)                          & \cellcolor[HTML]{FFCE93}0.301(0.026) \\
Logis  & 2.686(0.047)                         & \cellcolor[HTML]{FFCE93}0.449(0.004) & 0.27(0.002)                          \\
Cratio & 2.315(0.035)                         & 0.46(0.005)                          & 0.283(0.003)                         \\
acat   & 2.753(0.04)                          & 0.426(0.004)                         & 0.259(0.002)                         \\ \hline
\end{tabular}

\caption{模擬資料1維於B方法五種模型預測結果}\label{tab.5.2.1b}
\vspace{0.5cm}
}\end{subtable}
\hfill
\begin{subtable}[H]{.6\linewidth}
	\footnotesize
    \centering
    \extrarowheight=5pt
\setlength{\tabcolsep}{2mm}{
\begin{tabular}{cccc}
\hline
Model  & MSE                                  & Accuracy                             & Macro-F1                             \\ \hline
Clm    & \cellcolor[HTML]{FFCE93}1.359(0.165) & 0.471(0.025)                         & \cellcolor[HTML]{FFCE93}0.298(0.033) \\
Naive  & 1.955(0.459)                         & 0.444(0.042)                         & 0.247(0.031)                         \\
Logis  & 1.834(0.374)                         & 0.46(0.029)                          & 0.245(0.028)                         \\
Cratio & 1.925(0.263)                         & 0.411(0.026)                         & 0.256(0.032)                         \\
acat   & 1.686(0.202)                         & \cellcolor[HTML]{FFCE93}0.479(0.019) & 0.275(0.031)                         \\ \hline
\end{tabular}

    \caption{模擬資料2維於B方法五種模型預測結果}\label{tab.5.2.2b}
    \vspace{0.5cm}
}\end{subtable}
\hfill

\begin{subtable}[H]{.5\linewidth}
	\footnotesize
    \centering
    \extrarowheight=5pt
\setlength{\tabcolsep}{2mm}{
\begin{tabular}{cccc}
\hline
Model  & MSE                                 & Accuracy                             & Macro-F1                             \\ \hline
Clm    & \cellcolor[HTML]{FFCE93}1.981(0.09) & 0.571(0.015)                         & 0.282(0.006)                         \\
Naive  & 2.446(0.29)                         & 0.581(0.015)                         & 0.274(0.007)                         \\
Logis  & 2.297(0.177)                        & \cellcolor[HTML]{FFCE93}0.625(0.013) & 0.288(0.005)                         \\
Cratio & 2.36(0.103)                         & 0.506(0.018)                         & 0.261(0.006)                         \\
acat   & 2.151(0.094)                        & 0.621(0.014)                         & \cellcolor[HTML]{FFCE93}0.289(0.005) \\ \hline
\end{tabular}

\caption{模擬資料3維於B方法五種模型預測結果}\label{tab.5.2.3b}
\vspace{0.5cm}
}\end{subtable}
\hfill
\begin{subtable}[H]{.6\linewidth}
	\footnotesize
    \centering
    \extrarowheight=5pt
\setlength{\tabcolsep}{2mm}{
\begin{tabular}{cccc}
\hline
Model  & MSE                                  & Accuracy                             & Macro-F1                             \\ \hline
Clm    & \cellcolor[HTML]{FFCE93}1.726(0.022) & 0.516(0.004)                         & 0.356(0.003) \\
Naive  & 1.85(0.026)                          & 0.542(0.006)                         & 0.349(0.004)                         \\
Logis  & 1.856(0.027)                         & \cellcolor[HTML]{FFCE93}0.562(0.004) & 0.356(0.003) \\
Cratio & 1.934(0.024)                         & 0.482(0.005)                         & 0.34(0.003)                          \\
acat   & 1.804(0.023)                         & 0.551(0.004)                         & \cellcolor[HTML]{FFCE93}0.36(0.003)                          \\ \hline
\end{tabular}

    \caption{模擬資料4維於B方法五種模型預測結果}\label{tab.5.2.4b}
    \vspace{0.5cm}
}\end{subtable}

\begin{subtable}[H]{.5\textwidth}
	\footnotesize
    \centering
    \extrarowheight=5pt
\setlength{\tabcolsep}{2mm}{
\begin{tabular}{cccc}
\hline
Model  & MSE                                  & Accuracy                             & Macro-F1                             \\ \hline
Clm    & \cellcolor[HTML]{FFCE93}0.992(0.013) & 0.483(0.003)                         & \cellcolor[HTML]{FFCE93}0.438(0.003) \\
Naive  & 1.074(0.017)                         & 0.474(0.005)                         & 0.422(0.003)                         \\
Logis  & 1.05(0.013)                          & \cellcolor[HTML]{FFCE93}0.484(0.004) & 0.427(0.003)                         \\
Cratio & 1.084(0.015)                         & 0.459(0.004)                         & 0.42(0.003)                          \\
acat   & 1.049(0.013)                         & 0.483(0.004)                         & 0.427(0.003)                         \\ \hline
\end{tabular}

\caption{模擬資料5維於B方法五種模型預測結果}\label{tab.5.2.5b}
\vspace{0.5cm}
}\end{subtable}
\hfill
\begin{subtable}[H]{.6\linewidth}
	\footnotesize
    \centering
    \extrarowheight=5pt
\setlength{\tabcolsep}{2mm}{
\begin{tabular}{cccc}
\hline
Model  & MSE                                  & Accuracy                             & Macro-F1                             \\ \hline
Clm    & \cellcolor[HTML]{FFCE93}1.119(0.014) & 0.536(0.005)                         & \cellcolor[HTML]{FFCE93}0.414(0.003) \\
Naive  & 1.246(0.021)                         & 0.546(0.006)                         & 0.399(0.005)                         \\
Logis  & 1.181(0.016)                         & \cellcolor[HTML]{FFCE93}0.561(0.004) & 0.406(0.003)                         \\
Cratio & 1.277(0.016)                         & 0.499(0.005)                         & 0.392(0.003)                         \\
acat   & 1.159(0.014)                         & 0.558(0.004)                         & 0.413(0.003)                         \\ \hline
\end{tabular}

    \caption{模擬資料6維於B方法五種模型預測結果}\label{tab.5.2.6b}
    \vspace{0.5cm}
}\end{subtable}
\hfill
\begin{subtable}[H]{.5\linewidth}
	\footnotesize
    \centering
    \extrarowheight=5pt
\setlength{\tabcolsep}{2mm}{
\begin{tabular}{cccc}
\hline
Model  & MSE                                  & Accuracy                             & Macro-F1                             \\ \hline
Clm    & \cellcolor[HTML]{FFCE93}0.665(0.011) & 0.618(0.005)                         & \cellcolor[HTML]{FFCE93}0.432(0.005) \\
Naive  & 0.805(0.025)                         & 0.61(0.006)                          & 0.399(0.006)                         \\
Logis  & 0.698(0.015)                         & \cellcolor[HTML]{FFCE93}0.637(0.004) & 0.42(0.005)                          \\
Cratio & 0.801(0.017)                         & 0.575(0.006)                         & 0.401(0.005)                         \\
acat   & 0.702(0.013)                         & 0.619(0.005)                         & 0.42(0.005)                          \\ \hline
\end{tabular}

\caption{模擬資料7維於B方法五種模型預測結果}\label{tab.5.2.7b}
\vspace{0.5cm}
}\end{subtable}
\hfill
\begin{subtable}[H]{.6\linewidth}
	\footnotesize
    \centering
    \extrarowheight=5pt
\setlength{\tabcolsep}{2mm}{
\begin{tabular}{cccc}
\hline
Model  & MSE                                  & Accuracy                             & Macro-F1                             \\ \hline
Clm    & \cellcolor[HTML]{FFCE93}1.375(0.021) & 0.553(0.004)                         & 0.417(0.003)                         \\
Naive  & 1.637(0.038)                         & 0.547(0.006)                         & 0.402(0.004)                         \\
Logis  & 1.492(0.027)                         & \cellcolor[HTML]{FFCE93}0.588(0.004) & \cellcolor[HTML]{FFCE93}0.421(0.004) \\
Cratio & 1.457(0.023)                         & 0.538(0.004)                         & 0.411(0.003)                         \\
acat   & 1.388(0.021)                         & 0.567(0.004)                         & 0.418(0.003)                         \\ \hline
\end{tabular}

    \caption{模擬資料8維於B方法五種模型預測結果}\label{tab.5.2.8b}
    \vspace{0.5cm}
}\end{subtable}
\begin{subtable}[H]{.5\textwidth}
	\footnotesize
    \centering
    \extrarowheight=5pt
\setlength{\tabcolsep}{2mm}{
\begin{tabular}{cccc}
\hline
Model  & MSE                                  & Accuracy                             & Macro-F1                            \\ \hline
Clm    & 0.874(0.01)                          & 0.567(0.003)                         & \cellcolor[HTML]{FFCE93}0.48(0.004) \\
Naive  & \cellcolor[HTML]{FFCE93}0.987(0.017) & 0.562(0.004)                         & 0.461(0.004)                        \\
Logis  & 0.909(0.011)                         & \cellcolor[HTML]{FFCE93}0.584(0.003) & 0.475(0.003)                        \\
Cratio & 0.961(0.01)                          & 0.541(0.004)                         & 0.464(0.003)                        \\
acat   & 0.886(0.01)                          & 0.573(0.003)                         & 0.477(0.003)                        \\ \hline
\end{tabular}

\caption{模擬資料9維於B方法五種模型預測結果}\label{tab.5.2.9b}
\vspace{0.5cm}
}\end{subtable}
\hfill
\begin{subtable}[H]{.6\linewidth}
	\footnotesize
    \centering
    \extrarowheight=5pt
\setlength{\tabcolsep}{2mm}{
\begin{tabular}{cccc}
\hline
Model  & MSE                                  & Accuracy                             & Macro-F1                             \\ \hline
Clm    & \cellcolor[HTML]{FFCE93}1.228(0.017) & 0.564(0.004)                         & 0.421(0.003)                         \\
Naive  & 1.485(0.033)                         & 0.558(0.004)                         & 0.404(0.003)                         \\
Logis  & 1.303(0.021)                         & \cellcolor[HTML]{FFCE93}0.604(0.003) & \cellcolor[HTML]{FFCE93}0.425(0.004) \\
Cratio & 1.299(0.017)                         & 0.549(0.003)                         & 0.413(0.003)                         \\
acat   & 1.235(0.018)                         & 0.587(0.003)                         & 0.428(0.003)                         \\ \hline
\end{tabular}

    \caption{模擬資料10維於B方法五種模型預測結果}\label{tab.5.2.10b}
    \vspace{0.5cm}
}\end{subtable}
\end{table}


	由表\ref{table.5.2.A}與表\ref{table.5.2.B}中我們可以觀察到重抽樣建模與未重抽樣建模兩種情況於模擬資料之下的表現,整理出表\ref{tab.5.2.4}觀察到不論是名目型分類模型或是次序型分類模型,重抽樣資料建模(B方法)時,衡量指標Accuracy與Macro-F1的表現上皆比未重抽樣資料建模(A方法)時來得差,而MSE表現上則是各有勝負,從這部份我們猜測,因為這裡的重抽樣包含到欠採樣,意味著訓練資料的減少,也代表更難訓練出一個較為合適的模型。
	


\begin{table}[H]
	\footnotesize
    \centering
    \extrarowheight=5pt
    \caption{模擬資料於A、B兩種抽樣方法最佳模型表現差異}\label{tab.5.2.4}
\setlength{\tabcolsep}{3mm}{
\begin{threeparttable}
\begin{tabular}{cccccccccc}
\hline
                             & \multicolumn{3}{c}{MSE}                                              & \multicolumn{3}{c}{Accuracy}                                         & \multicolumn{3}{c}{Macro-F1}                    \\ \hline
                             & A方法   & B方法   & \multicolumn{1}{c|}{相差}                              & A方法   & B方法   & \multicolumn{1}{c|}{相差}                              & A方法   & B方法   & 相差                              \\ \hline
\multicolumn{1}{c|}{Data 1}  & 1.427 & 2.225 & \multicolumn{1}{c|}{{\color[HTML]{FE0000} 55.92\%}}  & 0.563 & 0.449 & \multicolumn{1}{c|}{{\color[HTML]{FE0000} -20.25\%}} & 0.413 & 0.301 & {\color[HTML]{FE0000} -27.12\%} \\ \hline
\multicolumn{1}{c|}{Data 2}  & 0.829 & 1.359 & \multicolumn{1}{c|}{{\color[HTML]{FE0000} 63.93\%}}  & 0.656 & 0.479 & \multicolumn{1}{c|}{{\color[HTML]{FE0000} -26.98\%}} & 0.665 & 0.298 & {\color[HTML]{FE0000} -55.19\%} \\ \hline
\multicolumn{1}{c|}{Data 3}  & 1.543 & 1.981 & \multicolumn{1}{c|}{{\color[HTML]{FE0000} 28.39\%}}  & 0.833 & 0.625 & \multicolumn{1}{c|}{{\color[HTML]{FE0000} -24.97\%}} & 0.666 & 0.289 & {\color[HTML]{FE0000} -56.61\%} \\ \hline
\multicolumn{1}{c|}{Data 4}  & 2.283 & 1.726 & \multicolumn{1}{c|}{{\color[HTML]{009901} -24.40\%}} & 0.697 & 0.562 & \multicolumn{1}{c|}{{\color[HTML]{FE0000} -19.37\%}} & 0.706 & 0.356 & {\color[HTML]{FE0000} -49.58\%} \\ \hline
\multicolumn{1}{c|}{Data 5}  & 1.048 & 0.992 & \multicolumn{1}{c|}{{\color[HTML]{009901} -5.34\%}}  & 0.599 & 0.484 & \multicolumn{1}{c|}{{\color[HTML]{FE0000} -19.20\%}} & 0.619 & 0.438 & {\color[HTML]{FE0000} -29.24\%} \\ \hline
\multicolumn{1}{c|}{Data 6}  & 1.235 & 1.119 & \multicolumn{1}{c|}{{\color[HTML]{009901} -9.39\%}}  & 0.674 & 0.561 & \multicolumn{1}{c|}{{\color[HTML]{FE0000} -16.77\%}} & 0.575 & 0.414 & {\color[HTML]{FE0000} -28.00\%} \\ \hline
\multicolumn{1}{c|}{Data 7}  & 0.518 & 0.665 & \multicolumn{1}{c|}{{\color[HTML]{FE0000} 28.38\%}}  & 0.73  & 0.637 & \multicolumn{1}{c|}{{\color[HTML]{FE0000} -12.74\%}} & 0.562 & 0.432 & {\color[HTML]{FE0000} -23.13\%} \\ \hline
\multicolumn{1}{c|}{Data 8}  & 2.061 & 1.375 & \multicolumn{1}{c|}{{\color[HTML]{009901} -33.28\%}} & 0.678 & 0.588 & \multicolumn{1}{c|}{{\color[HTML]{FE0000} -13.27\%}} & 0.765 & 0.421 & {\color[HTML]{FE0000} -44.97\%} \\ \hline
\multicolumn{1}{c|}{Data 9}  & 1.095 & 0.987 & \multicolumn{1}{c|}{{\color[HTML]{009901} -9.86\%}}  & 0.647 & 0.584 & \multicolumn{1}{c|}{{\color[HTML]{FE0000} -9.74\%}}  & 0.639 & 0.48  & {\color[HTML]{FE0000} -24.88\%} \\ \hline
\multicolumn{1}{c|}{Data 10} & 1.963 & 1.228 & \multicolumn{1}{c|}{{\color[HTML]{009901} -37.44\%}} & 0.709 & 0.604 & \multicolumn{1}{c|}{{\color[HTML]{FE0000} -14.81\%}} & 0.788 & 0.425 & {\color[HTML]{FE0000} -46.07\%} \\ \hline
\end{tabular}

\begin{tablenotes}  
        \item[1.] :  紅色代表A方法優於B方法,綠色反之
        \item[2.] :  相差的計算方法為: (B方法 - A方法) / A方法
\end{tablenotes}
\end{threeparttable}
}\end{table}


\section{文字資料應用}
	
	本節我們使用Yahoo國片電影評論資料與Kaggle平台上所提供的Trip Advisor Hotel Reviews資料,透過文字預處理與詞嵌入的方法處理成模型能輸入的資料樣態後檢視模型於文字資料的表現,文字預處理的部分在第4.2章節有介紹。
	
	在上述兩節中,我們觀察到在符合模型假設之下,次序型資料中使用次序型分類模型之表現優於名目型分類模型,且未經過重抽樣建模的方法所配適出來的模型優於以低採樣方料重抽樣後的模型,因此在本章節我們將使用資料未經過重抽樣的方法來建模。在文字資料中訓練集與測試集與前面相同皆為7:3,且兩者皆依各類別等比例抽樣。
	
	在此我們將觀察小維度至大維度時模型預測結果的差異,考慮以下幾種不同的詞嵌入方法來衡量模型表現,其中CBOW與Skip-gram法皆使用負採樣(negative sampling)為5,令D表示維度,W(Window size)表示CBOW與Skip-gram法中獲取前後文單詞的數量,如下所示共有21組:
	
	
\begin{enumerate}[1.]
\setlength{\itemsep}{-10pt}
\item TF-IDF法 D = 50, D = 100, D = 150, D = 200
\item CBOW法 W = 3 ; D = 50, D = 100, D = 150, D = 200
\item CBOW法 W = 5 ; D = 50, D = 100, D = 150, D = 200
\item Skip-gram法 W = 3 ; D = 50, D = 100, D = 150, D = 200
\item Skip-gram法 W = 5 ; D = 50, D = 100, D = 150, D = 200
\item wiki法 D = 300

\end{enumerate}
	
	
% 中文
\subsection{中文文字-Yahoo電影評論}
	
	在此章節將以中文文字資料-Yahoo電影評論比較三種次序型分類模型:Cumulative Logit Model、Continuation-Ratio Logit Model、Adjacent-Category Logit Model,與兩種名目型分類模型:Naïve Bayes、多元邏輯斯模型於不同詞嵌入方法的表現,衡量指標包括MSE、Accuracy與Macro-F1。如圖4.1所示,本資料即之評分絕大多數為5分,極度不平衡。
	
\subsubsection{CBOW法於五種模型表現}

	首先,在MSE、Accuracy與Macro-F1指標上,同個分類模型在W=3和5兩種狀況下的表現相近,且於不同維度有著相似的趨勢,以圖\ref{pic.5.3.1-1}為例,我們可以看到CBOW法套用於五種模型時,次序型分類模型在Macro-F1的表現最佳,名目型分類模型上表現較差,而在次序型分類模型上又以維度D=100時表現最好,在維度D=50與200時表現略差。	
	
\newpage

\begin{figure}[H]
    \centering
        \includegraphics[scale=0.5]{\imgdir CBOW_F1.jpeg}
    \caption{CBOW法各模型於不同維度下Macro-F1表現-中文資料}
    \label{pic.5.3.1-1}
    \caption*{\footnotesize{Note: nnet套件中多元邏輯斯模型(nnet:: multinom)在維度200無法執行}}
\end{figure}
	

	再來以Accuracy指標來看CBOW法,各模型在W=3和5兩種狀況下,於不同維度趨勢依舊相近,由圖\ref{pic.5.3.1-2}可以觀察到名目型分類模型Naïve Bayes的表現最差,其餘四種模型表現相當一致。
	
\begin{figure}[H]
    \centering
        \includegraphics[scale=0.5]{\imgdir CBOW_ACC.jpeg}
    \caption{CBOW法各模型於不同維度下Accuracy表現-中文資料}
    \label{pic.5.3.1-2}
    \caption*{\footnotesize{Note: nnet套件中多元邏輯斯模型(nnet:: multinom)在維度200無法執行}}
\end{figure}	

	而從MSE指標來看,如圖\ref{pic.5.3.1-3}依舊是名目型分類模型Naïve Bayes的表現最差,其餘四種分類模型表現相當一致,而多元邏輯斯模型的表現在MSE指標上是表現得較好的,以維度D=100表現最佳。
	
\begin{figure}[H]
    \centering
        \includegraphics[scale=0.5]{\imgdir CBOW_MSE.jpeg}
    \caption{CBOW法各模型於不同維度下MSE表現-中文資料}
    \label{pic.5.3.1-3}
    \caption*{\footnotesize{Note: nnet套件中多元邏輯斯模型(nnet:: multinom)在維度200無法執行}}
\end{figure}	
	
	從上面我們觀察到,雖然多元邏輯斯模型在Accuracy有著較好且接近次序型分類模型的表現,但在Macro-F1上卻遠遠低於次序型分類模型,由此可見多元邏輯斯模型在本資料集為不平衡資料的情況下,相較於次序型分類模型容易傾向預測資料筆數較多的一類,因此在Accuracy上表現較佳,Macro-F1表現較差。
	
	
\subsubsection{Skip-gram法於五種模型表現}

	首先如同CBOW法,在Skip-gram法中,在MSE、Accuracy與Macro-F1各指標上,同個模型在W=3和5兩種狀況下,於不同維度有著相似的趨勢,以圖\ref{pic.5.3.2-1}為例,我們可以看到Skip-gram法套用於五種模型時,次序型分類模型在Macro-F1的表現最佳,名目型分類模型上表現較差,而在次序型分類模型上又以維度D=50與100時表現較好,在維度D=200時表現略差。
	
\begin{figure}[h]
    \centering
        \includegraphics[scale=0.5]{\imgdir SKIP_F1.jpeg}
    \caption{Skip-gram法各模型於不同維度下Macro-F1表現-中文資料}
    \label{pic.5.3.2-1}
    \caption*{\footnotesize{Note: nnet套件中多元邏輯斯模型(nnet:: multinom)在維度200無法執行}}
\end{figure}
	
	再來以Accuracy指標來看Skip-gram法時,各模型在W=3和5兩種狀況下於不同維度趨勢依舊相近,由圖\ref{pic.5.3.2-2}可以觀察到名目型分類模型Naïve Bayes的表現最差,其餘四種模型表現相當一致。
	
\begin{figure}[H]
    \centering
        \includegraphics[scale=0.5]{\imgdir SKIP_ACC.jpeg}
    \caption{Skip-gram法各模型於不同維度下Accuracy表現-中文資料}
    \label{pic.5.3.2-2}
    \caption*{\footnotesize{Note: nnet套件中多元邏輯斯模型(nnet:: multinom)在維度200無法執行}}
\end{figure}	
	
	而從MSE指標來看,圖\ref{pic.5.3.2-3}顯示Naïve Bayes於W=3時的表現最差,多元邏輯斯模型則是在W=5且D=100時表現最佳,值得一提的是次序型分類模型與多元邏輯斯模型在D=100時表現都較好,但隨著維度越大,MSE也放大許多,甚至高過Naïve Bayes。
	
\begin{figure}[H]
    \centering
        \includegraphics[scale=0.4]{\imgdir SKIP_MSE.jpeg}
    \caption{Skip-gram法各模型於不同維度下MSE表現-中文資料}
    \label{pic.5.3.2-3}
    \caption*{\footnotesize{Note: nnet套件中多元邏輯斯模型(nnet:: multinom)在維度200無法執行}}
\end{figure}	
	
	從上面我們觀察到Skip-gram與CBOW法的結果相似,多元邏輯斯模型在本資料集為不平衡資料的情況下,預測結果相較於次序型分類模型容易傾向資料筆數較多的一類,因此在Accuracy上表現較佳,Macro-F1表現較差,而Skip-gram法與CBOW法模型大多在維度D=100時表現較其它維度好。


\subsubsection{TF-IDF法於五種模型表現}

	如圖\ref{pic.5.3.3-1}所示,我們可以看到TF-IDF法套用於五種模型時Macro-F1的表現上,次序型分類模型的表現都最佳,名目型分類模型上表現較差,而在次序型分類模型上又以維度D=150與200時表現最好,在維度D=50時表現最不佳,三種模型在不同維度時的趨勢相似。
	
\begin{figure}[H]
    \centering
        \includegraphics[scale=0.4]{\imgdir TFIDF_F1.jpeg}
    \caption{TF-IDF法各模型於不同維度下Macro-F1表現-中文資料}
    \label{pic.5.3.3-1}
    \caption*{\footnotesize{Note: nnet套件中多元邏輯斯模型(nnet:: multinom)在維度200無法執行}}
\end{figure}
	
	
\newpage


	再來以Accuracy與MSE指標來看TF-IDF法,由圖\ref{pic.5.3.3-2}與圖5.9可以觀察到名目型分類模型Naïve Bayes的表現最差,其餘四種模型表現相近,且在維度D=150與200時表現較佳。
	
\begin{figure}[H]
    \centering
        \includegraphics[scale=0.5]{\imgdir TFIDF_ACC.jpeg}
    \caption{TF-IDF法各模型於不同維度下Accuracy表現-中文資料}
    \label{pic.5.3.3-2}
    \caption*{\footnotesize{Note: nnet套件中多元邏輯斯模型(nnet:: multinom)在維度200無法執行}}
\end{figure}	
	
	
\begin{figure}[H]
    \centering
        \includegraphics[scale=0.5]{\imgdir TFIDF_MSE.jpeg}
    \caption{TF-IDF法各模型於不同維度下MSE表現-中文資料}
    \label{pic.5.3.3-3}
    \caption*{\footnotesize{Note: nnet套件中多元邏輯斯模型(nnet:: multinom)在維度200無法執行}}
\end{figure}	
	
	從上面我們可以得到一個結論,與CBOW法和Skip-gram法發生的狀況一樣,多元邏輯斯模型在本資料集為不平衡資料的情況下,預測結果相較於次序型分類模型容易傾向資料筆數較多的一邊,因此在Accuracy上表現較佳,Macro-F1表現較差,而與Skip-gram法和CBOW法模型大多在低維度D=100時表現較好之下,TF-IDF法反而是在維度越高D=150與200時表現較好。
	
\subsubsection{wiki法於五種模型表現}
	
	wiki法維度皆為300維,由表\ref{tab.5.3.4}中可以看到wiki法在次序型分類模型有著較好的表現,但相較CBOW法、Skip-gram法和TF-IDF法並不會特別出色,因本研究使用的多元邏輯斯模型無法於高維度執行,故未列出該模型於wiki法的成效。
	
\begin{table}[H]
	\footnotesize
    \centering
    \extrarowheight=5pt
    \caption{Wiki法於五種分類模型之比較-中文資料}\label{tab.5.3.4}
\setlength{\tabcolsep}{5mm}{
\begin{tabular}{cccc}
\hline
       & MSE          & Accuracy     & Macro-F1     \\ \hline
Clm.   & 2.454(0.135) & 0.737(0.009) & 0.71(0.015)  \\ 
cratio & 2.484(0.153) & 0.735(0.01)  & 0.706(0.016) \\ 
acat   & 2.479(0.152) & 0.735(0.01)  & 0.71(0.017)  \\ 
Naïve  & 4.403(0.35)  & 0.267(0.03)  & 0.194(0.027) \\ \hline
\end{tabular}
\begin{tablenotes} 
\scriptsize 
        \item[1.] \;\;\;\;\;\;\;\;\;\;\;\;\;\;\;\;\;\;\;\;\;\;\;\;\;\;\;\;nnet套件中多元邏輯斯模型(nnet:: multinom)在維度300無法執行
       
\end{tablenotes}
}\end{table}
	
	
\newpage

\subsubsection{綜合評估}

	下面我們將列出模型綜合比較表,可以觀察到這筆資料集在使用skip-gram時表現較好,且Adjacent-Category Logit Model搭配以Skip-gram詞嵌入法當W=5且維度D=50時有最佳表現。

\begin{table}[H]
	\footnotesize
    \centering
    \extrarowheight=5pt
    \caption{模型綜合比較表-中文資料}\label{tab.5.3.5}
\setlength{\tabcolsep}{5mm}{
\begin{tabular}{ccccc}
\hline
\textbf{模型}               & \textbf{文字特徵}       & \textbf{MSE} & \textbf{Accuracy} & \textbf{Macro-F1} \\ \hline
\multirow{3}{*}{Clm}      & CBOW W=5 D=100      & 2.014(0.129) & 0.761(0.008)      & 0.74(0.014)       \\
                          & Skip-gram W=5 D=100 & 1.76(0.144)  & 0.776(0.009)      & 0.771(0.016)      \\
                          & TF-IDF D=200        & 2.481(0.131) & 0.734(0.008)      & 0.689(0.015)      \\ \cdashline{1-5}[0.8pt/4pt]
\multirow{3}{*}{Cratio}   & CBOW W=5 D=150      & 2.107(0.149) & 0.756(0.009)      & 0.734(0.017)      \\
                          & Skip-gram W=5 D=100 & 1.78(0.142)  & 0.775(0.008)      & 0.766(0.015)      \\
                          & TF-IDF D=200        & 2.497(0.133) & 0.733(0.008)      & 0.684(0.016)      \\
\cdashline{1-5}[0.8pt/4pt]
\multirow{3}{*}{acat}     & CBOW W=5 D=100      & 2.01(0.131)  & 0.761(0.008)      & 0.743(0.014)      \\
                          & Skip-gram W=5 D=50  & 1.726(0.154) & 0.779(0.009)      & 0.778(0.018)      \\
                          & TF-IDF D=150        & 2.404(0.147) & 0.737(0.009)      & 0.691(0.018)      \\
\cdashline{1-5}[0.8pt/4pt]
\multirow{3}{*}{Naïve}    & CBOW W=5 D=50       & 2.577(0.245) & 0.441(0.02)       & 0.268(0.018)      \\
                          & Skip-gram W=5 D=100 & 1.695(0.141) & 0.54(0.019)       & 0.337(0.018)      \\
                          & TF-IDF D=50         & 4.677(0.442) & 0.257(0.055)      & 0.173(0.031)      \\
\cdashline{1-5}[0.8pt/4pt]
\multirow{3}{*}{Logistic} & CBOW W=3 D=50       & 2.114(0.124) & 0.748(0.008)      & 0.391(0.051)      \\
                          & Skip-gram W=5 D=150 & 1.635(0.106) & 0.754(0.013)      & 0.393(0.026)      \\
                          & TF-IDF D=150        & 2.305(0.126) & 0.714(0.011)      & 0.37(0.021)       \\ \hline
\end{tabular}
}\end{table}


	在本資料集中我們可以得到以下幾點結論:
	
\begin{enumerate}[1.]
\setlength{\itemsep}{-10pt}
\item \text{不論是在CBOW法或是Skip-gram法中,W=3或5對結果影響不大。}
\item \text{整體而言次序型分類模型較名目型分類模型表現較佳}
\item \text{整體而言Skip-gram法最佳,CBOW法與wiki法次之,最後是TF-IDF法}
\end{enumerate}


%%%%%%%%%%%%%%%%%%%%%%%%%%%%%%%%%%%%%%%%%%%%%%%%%%%%%%%%%%
% 英文
\subsection{英文文字-Trip Advisor Hotel Reviews}
	
	在此章節將以英文文字資料-Trip Advisor Hotel Reviews展示三種次序型分類模型:Cumulative Logit Model、Continuation-Ratio Logit Model、Adjacent-Category Logit Model,與兩種名目型分類模型:Naïve Bayes、多元邏輯斯模型於不同詞嵌入方法的表現,衡量指標包括MSE、Accuracy與Macro-F1。如圖4.2所示,本資料即之評分絕大多數為5分,極度不平衡。
	
\subsubsection{CBOW法於五種模型表現}

	首先,在MSE、Accuracy與Macro-F1指標上,同個模型在W=3或5兩種狀況下於不同維度有著相似的趨勢,以圖\ref{pic.5.3.6-1}為例,我們看到CBOW法套用於五種模型時,多元邏輯斯模型在Macro-F1上表現較次序型分類模型佳,且隨著維度增加,Macro-F1也隨之上升,反倒是過去一直表現不錯的Adjacent-Category Logit Model在維度D=150時有個向下V型跌落,與其他模型於維度D=150時有著最佳表現相反。
		
\begin{figure}[H]
    \centering
        \includegraphics[scale=0.5]{\imgdir CBOW_F11.jpeg}
    \caption{CBOW法各模型於不同維度下Macro-F1表現-英文資料}
    \label{pic.5.3.6-1}
    \caption*{\footnotesize{Note: nnet套件中多元邏輯斯模型(nnet:: multinom)在維度200無法執行}}
\end{figure}
	
	再來以Accuracy指標來看CBOW法時,各模型在W=3和5兩種狀況下於不同維度趨勢依舊相近,由圖\ref{pic.5.3.6-2}可以觀察到多元邏輯斯模型的表現最佳,次序型分類模型其次,再來是Naïve Bayes,其中除了Naïve Bayes在維度D=150時表現最差,其餘模型皆在該維度有著最好的表現。
	
	然而這五種模型的Accuracy大概介於0.49至0.62之間,差距並不會到非常大,且各模型在W=5時Accuracy皆比W=3時佳。
	
\begin{figure}[H]
    \centering
        \includegraphics[scale=0.5]{\imgdir CBOW_ACC1.jpeg}
    \caption{CBOW法各模型於不同維度下Accuracy表現-英文資料}
    \label{pic.5.3.6-2}
    \caption*{\footnotesize{Note: nnet套件中多元邏輯斯模型(nnet:: multinom)在維度200無法執行}}
\end{figure}	
	
	而從MSE指標來看,如圖\ref{pic.5.3.6-3}依舊是Naïve Bayes的表現最差,其次為次序型分類模型,多元邏輯斯模型的表現在MSE指標上是表現得較好的,以維度D=100表現最佳,而這邊也能看出各模型在W=5時表現較W=3時好。
	
\begin{figure}[H]
    \centering
        \includegraphics[scale=0.5]{\imgdir CBOW_MSE1.jpeg}
    \caption{CBOW法各模型於不同維度下MSE表現-英文資料}
    \label{pic.5.3.6-3}
    \caption*{\footnotesize{Note: nnet套件中多元邏輯斯模型(nnet:: multinom)在維度200無法執行}}
\end{figure}	
	
	從上面我們可以得到一個結論,多元邏輯斯模型在CBOW法中最佳、最穩定且在維度D=150時有最好的表現;而各模型在W=5時的表現都比W=3時好。
	
	
\subsubsection{Skip-gram法於五種模型表現}

	在Skip-gram法中,在Macro-F1指標上,Adjacent-Category Logit Model在維度D=100與150時的表現相較維度D=50與200時突然下降,其餘則與在CBOW法中相似,以多元邏輯斯模型表現最佳。
		
\begin{figure}[H]
    \centering
        \includegraphics[scale=0.4]{\imgdir SKIP_F11.jpeg}
    \caption{Skip-gram法各模型於不同維度下Macro-F1表現-英文資料}
    \label{pic.5.3.7-1}
    \caption*{\footnotesize{Note: nnet套件中多元邏輯斯模型(nnet:: multinom)在維度200無法執行}}
\end{figure}
	
	Skip-gram法中Accuracy與MSE指標的表現與CBOW法相似,以多元邏輯斯模型表現最佳,其次為次序型分類模型,最後是Naïve Bayes。
	
\begin{figure}[H]
    \centering
        \includegraphics[scale=0.4]{\imgdir SKIP_ACC1.jpeg}
    \caption{Skip-gram法各模型於不同維度下Accuracy表現-英文資料}
    \label{pic.5.3.7-2}
    \caption*{\footnotesize{Note: nnet套件中多元邏輯斯模型(nnet:: multinom)在維度200無法執行}}
\end{figure}	
	
\begin{figure}[H]
    \centering
        \includegraphics[scale=0.4]{\imgdir SKIP_MSE1.jpeg}
    \caption{Skip-gram法各模型於不同維度下MSE表現-英文資料}
    \label{pic.5.3.7-3}
    \caption*{\footnotesize{Note: nnet套件中多元邏輯斯模型(nnet:: multinom)在維度200無法執行}}
\end{figure}	
	
	從上面我們可以得到一個結論,與CBOW法發生的狀況一樣,多元邏輯斯模型在本資料集接表現的最佳,而過去表現的較好的Adjacent-Category Logit Model在各維度有著不穩定的狀況;且各模型在W=5時表現皆比W=3時好。




\subsubsection{TF-IDF法於五種模型表現}

	如圖\ref{pic.5.3.8-1}所示,與前面其他詞嵌入方法的結果不同,在這邊各模型在Macro-F1於各維度表現並不一致,可以看到次序型分類模型沒有一個固定的趨勢,在各維度上的表現難以捉摸;而名目型分類模型皆隨著維度增加表現也隨之上升。
	
\begin{figure}[H]
    \centering
        \includegraphics[scale=0.4]{\imgdir TFIDF_F11.jpeg}
    \caption{TF-IDF法各模型於不同維度下Macro-F1表現-英文資料}
    \label{pic.5.3.8-1}
    \caption*{\footnotesize{Note: nnet套件中多元邏輯斯模型(nnet:: multinom)在維度200無法執行}}
\end{figure}
	
	再來以Accuracy指標來看TF-IDF法,由圖\ref{pic.5.3.8-2}可以觀察到多元邏輯斯模型表現最佳,其次為次序型分類模型,Naïve Bayes的表現最差,這五種模型皆隨著維度增加,Accuracy表現也隨之提升,當維度D=200時次序型分類模型的正確率已經接近多元邏輯斯模型維度D=150時的正確率。
	
\begin{figure}[H]
    \centering
        \includegraphics[scale=0.5]{\imgdir TFIDF_ACC1.jpeg}
    \caption{TF-IDF法各模型於不同維度下Accuracy表現-英文資料}
    \label{pic.5.3.8-2}
    \caption*{\footnotesize{Note: nnet套件中多元邏輯斯模型(nnet:: multinom)在維度200無法執行}}
\end{figure}	
	
	而從MSE指標來看,如圖\ref{pic.5.3.8-3}顯示Naïve Bayes的表現最差,其餘模型表現相近,且這個五種模型皆隨維度增加,MSE也隨之降低。
	
\begin{figure}[H]
    \centering
        \includegraphics[scale=0.5]{\imgdir TFIDF_MSE1.jpeg}
    \caption{TF-IDF法各模型於不同維度下MSE表現-英文資料}
    \label{pic.5.3.8-3}
    \caption*{\footnotesize{Note: nnet套件中多元邏輯斯模型(nnet:: multinom)在維度200無法執行}}
\end{figure}	
	
	從上面我們可以得到一個結論,與CBOW法和Skip-gram法發生的狀況一樣,多元邏輯斯模型的表現最佳,次序型分類模型在Macro-F1上於各維度表現不穩定,並無法判斷出該模型最適合的維度,相較於CBOW法和Skip-gram法皆在維度D=150時表現較其他維度好,TF-IDF法則是在維度D=200時有較好的表現。
	
\subsubsection{wiki法於五種模型表現}

	wiki法維度皆為300維,由表\ref{tab.5.3.9}中可以看到wiki法在Naïve Bayes有著較好的表現,但相較CBOW法、Skip-gram法和TF-IDF法並不會特別出色,因本研究使用的多元邏輯斯模型無法於高維度執行,故未列出該模型於wiki法的成效。
		
\begin{table}[H]
	\footnotesize
    \centering
    \extrarowheight=5pt
    \caption{Wiki法於五種分類模型之比較-英文資料}\label{tab.5.3.9}
\setlength{\tabcolsep}{5mm}{
\begin{tabular}{cccc}
\hline
       & MSE          & Accuracy     & Macro-F1     \\ \hline
Clm.   & 0.7(0.011)   & 0.568(0.005) & 0.45(0.006)  \\ 
cratio & 0.712(0.01)  & 0.567(0.005) & 0.443(0.006) \\ 
acat   & 0.698(0.012) & 0.57(0.005)  & 0.439(0.006)  \\ 
Naïve  & 1.286(0.03)  & 0.494(0.006) & 0.453(0.007) \\ \hline
\end{tabular}
\begin{tablenotes}
  \scriptsize
        \item[1.] \;\;\;\;\;\;\;\;\;\;\;\;\;\;\;\;\;\;\;\;\;\;\;\;\;\;\;\;nnet套件中多元邏輯斯模型(nnet:: multinom)在維度300無法執行
       
\end{tablenotes}
}\end{table}

\newpage

\subsubsection{綜合評估}


	下面我們將列出模型綜合比較表,可以觀察到這筆資料集在多元邏輯斯模型搭配以CBOW詞嵌入法當W=5且維度D為150時有最佳表現。

\begin{table}[H]
	\footnotesize
    \centering
    \extrarowheight=5pt
    \caption{模型綜合比較表}\label{tab.5.3.5}
\setlength{\tabcolsep}{5mm}{
\begin{tabular}{ccccc}
\hline
\textbf{模型}               & \textbf{文字特徵}       & \textbf{MSE} & \textbf{Accuracy} & \textbf{Macro-F1} \\ \hline
\multirow{3}{*}{Clm}      & CBOW W=5 D=150      & 0.699(0.013) & 0.568(0.005)      & 0.449(0.006)      \\
                          & Skip-gram W=5 D=150 & 0.658(0.013) & 0.572(0.004)      & 0.466(0.006)      \\
                          & TF-IDF D=200        & 0.742(0.012) & 0.561(0.004)      & 0.436(0.007)      \\ \cdashline{1-5}[0.8pt/4pt]
\multirow{3}{*}{Cratio}   & CBOW W=5 D=150      & 0.71(0.012)  & 0.567(0.004)      & 0.443(0.005)      \\
                          & Skip-gram W=5 D=100 & 0.672(0.011) & 0.572(0.004)      & 0.461(0.006)      \\
                          & TF-IDF D=200        & 0.735(0.013) & 0.564(0.004)      & 0.507(0.043)      \\\cdashline{1-5}[0.8pt/4pt]
\multirow{3}{*}{acat}     & CBOW W=3 D=100      & 0.755(0.013) & 0.558(0.004)      & 0.521(0.008)      \\
                          & Skip-gram W=5 D=50  & 0.739(0.013) & 0.562(0.004)      & 0.518(0.029)      \\
                          & TF-IDF D=200        & 0.735(0.013) & 0.564(0.004)      & 0.507(0.043)      \\\cdashline{1-5}[0.8pt/4pt]
\multirow{3}{*}{Naïve Bayes}    & CBOW W=5 D=50       & 1.105(0.028) & 0.514(0.006)      & 0.46(0.007)       \\
                          & Skip-gram W=5 D=200 & 1.104(0.029) & 0.515(0.006)      & 0.46(0.007)       \\
                          & TF-IDF D=200        & 1.262(0.027) & 0.497(0.005)      & 0.453(0.006)      \\\cdashline{1-5}[0.8pt/4pt]
\multirow{3}{*}{Logistic} & CBOW W=5 D=150      & 0.617(0.014) & 0.619(0.005)      & 0.545(0.007)      \\
                          & Skip-gram W=5 D=150 & 0.614(0.014) & 0.615(0.005)      & 0.538(0.007)      \\
                          & TF-IDF D=150        & 1.015(0.027) & 0.57(0.005)       & 0.469(0.007)      \\ \hline
\end{tabular}
}\end{table}


	在本資料集中我們可以得到以下幾點結論:
	
\begin{enumerate}[1.]
\setlength{\itemsep}{-10pt}
\item \text{不論是在CBOW法或是Skip-gram法中,W=3或5對結果影響不大}
\item \text{整體而言多元邏輯斯模型表現最佳,次序型分類模型次之,最後是Naïve Bayes}
\item \text{詞嵌入方法於各模型表現有好有壞,無法判斷出最佳的詞向量方法}
\item \text{次序型分類模型在TF-IDF法時於各維度有著不穩定表現,而Adjacent-Category Logit }\\	\text{Model於另外兩種詞嵌入方法表現也較不穩定}
\end{enumerate}



	
\newpage
%\end{document}











%\input{preamble1}
%\usepackage{wallpaper}                                          % 使用浮水印
%\CenterWallPaper{0.6}{images/ntpu.eps}                           % 浮水印圖檔
%\begin{document}
\fontsize{12}{22pt}\selectfont
\cleardoublepage
\thispagestyle{empty}
\setlength{\parindent}{2em}
\chapter{研究結論與建議}

	本章節將會對實驗過程與實驗結果做出總結,實驗過程依序包含對數值與類別型資料、模擬資料與中英文文字資料,最後將給出研究建議。
	
\section{研究結論}

	首先,在三種次序型分類模型Cumulative Logit Model、Continuation-Ratio Logit Model、Adjacent-Category Logit Model與兩種名目型分類模型Naïve Bayes、多元邏輯斯模型,共五種模型之下,我們預測13組目標變數為次序型的數值與類別型態資料,發現名目型分類模型與次序型分類模型的表現不分軒輊,名目型分類模型以多元邏輯斯模型表現最好,次序型分類模型以Cumulative Logit Model最好,而Naïve Bayes在所有資料集中都表現不好,名目型與次序型分類模型表現不分上下的原因可能在於雖然資料的目標變數定義上為次序型,但資料本身並沒有次序型分類模型的比例賠率假設(Proportional Odds Assumption)。透過比例賠率假設檢定,我們發現這13組資料集中。較多變數符合此假設的資料集,使用次序型分類模型中有較好的成效。
	
	過去傳統統計模型強調「解釋能力」(配適能力),若符合前提假設可以預期模型有良好的解釋能力,但較無探討預測能力,因此在5.2章我們使用統計模擬去驗證是否符合前提假設之下的資料在次序型分類模型同樣會有較好的預測能力,結果發現確實這十組模擬資料在次序型分類模型上的成效大多優於名目型分類模型,唯二在名目型分類模型上表現較好的資料集於三種衡量指標上皆與次序型分類模型差距非常小。由上述可以知道,即使資料本身定義為次序型資料,但若其資料樣態並不滿足比例賠率假設,使用次序型分類模型未必獲得較佳預測結果。反之,若資料滿足該假設,則使用次序型分類模型通常有較佳表現。
	
	因現實中資料經常存在資料不平衡的情況,在本研究中也有對此做討論,分為A、B兩種情況,分別為未重抽樣資料建模與低採樣重抽樣資料建模,結果發現不論是在次序型分類模型或名目型分類模型時,皆在資料未經重抽樣之下建模有著較好的預測能力,原因可能在於重抽樣意味著訓練資料的減少,即使重抽樣後的資料筆數仍然具有5000筆以上的訓練資料集,同樣也較難配適出一個更合適的模型。

	應用於文字的部分不論是在中文或英文評論上,詞嵌入的部分CBOW法與Skip-gram法表現最佳,其次是wiki法,最後則是TF-IDF法,由此我們猜測CBOW法與Skip-gram法因考量到當前文本前後文的關聯性,所以轉換為詞向量後的資訊相較其他方法更為貼近文字本身所傳達的意思,且從參數W得知考量到更多前後文確實會提升預測能力。然而這兩種詞向量訓練方法耗時,若能直接使用網路上別人訓練好的wiki法較為省時且也有不差的預測結果,而維度(D)則是在100到150之間有著最佳結果,維度選取上可以在對於文本有更多了解後做文字特徵處理以選取合邏輯下的最佳維度。
	
	中文評論使用次序型分類模型有著較好的預測結果,英文評論使用多元邏輯斯模型有著較好的預測結果,Naïve Bayes則是皆表現最差,過去國外自然語言處理領域中,大多使用多元邏輯斯模型搭配詞嵌入來做預測,本研究發現在中文文字上使用次序型分類模型相較名目型分類模型有著較好的結果,中文與英文本身語句邏輯就不同,或許未來可朝向這部分更深入探究其原因。
	
\newpage

\section{研究建議}

	比例賠率假設是較為嚴格的檢定,變數與資料量大時容易拒絕,在文字使用詞嵌入法轉為詞向量後的維度通常也較高,訓練模型在高維度特徵時容易產生過擬合(Overfitting)的問題,若單純把詞向量中某幾個變數移除可能會影響詞向量本身表達的意思,在已知次序型分佈的資料使用對應的模型時會有較好結果之下,未來可在詞向量特徵選取,或使用正規化(regulariztion)以避免過擬合等議題上做更多的研究。
	
	本研究中使用的多元邏輯斯模型取自nnet套件,在維度為200以上無法執行,未來可嘗試其他套件以應用於更多情況;本研究中使用的Naïve Bayes取自e1071套件,在本研究中的表現相較其餘模型皆較差,或許因為資料集不滿足Naïve Bayes的變數獨立假設,此議題可於未來加以探討。
	
	本研究中重抽樣方法使用最為常見的過採樣(Over Sampling)與欠採樣(Undersampling),而此方法得到較差的預測結果,未來可考量其他處理不平衡資料的方法來改進資料不平衡的問題,重抽樣方法如4.1.1章節所述。
	
	本研究使用MSE做為參考預測與實際結果等級差異的指標,MSE算法是$l2$正則化,會將預測差距較大的給予更大的懲罰項,未來可使用MAE,也就是$l1$正則化做為衡量指標。



%\end{document}




\nocite{*}               % 秀出 .bib 文獻檔中的所有文獻,若拿掉這個命令,只秀出文章所用的文獻
%%%%%%%%%%% Reference %%%%%%%%%%%%%%%%%%%%%%
%\ifodd\count0 \else \thispagestyle{plain}\mbox{}\clearpage \fi

\addcontentsline{toc}{chapter}{參考文獻}   % 在目錄中添加"參考文獻"的
%\input{preamble1}
%\usepackage{wallpaper}                                          % 使用浮水印
%\CenterWallPaper{0.6}{images/ntpu.eps}                           % 浮水印圖檔
%\begin{document}
\fontsize{12}{22pt}\selectfont
\cleardoublepage
\thispagestyle{empty}
\setlength{\parindent}{2em}
\chapter*{參考文獻}

\noindent Alan Agresti(2003). \textit{Categorical Data Analysis 3rd Edition}, A JOHN WILEY $\&$ SONS, INC., PUBLICATION.

\noindent Jones, K. S.(1972). A statistical interpretation of term specificity and its application in retrieval. \textit{Journal of documentation}.

\noindent Liu, B.(2020). Text sentiment analysis based on CBOW model and deep learning in big data environment. \textit{Journal of Ambient Intelligence and Humanized Computing}, $\bm{11(2)}$, 451-458.

\noindent McCullagh, P.(1980). Regression models for ordinal data.\textit{Journal of the Royal Statistical Society: Series B (Methodological)}, $\bm{42(2)}$, 109-127.

\noindent Cardoso, J., $\&$ da Costa, J. P.(2007). Learning to Classify Ordinal Data: The Data Replication Method. \textit{Journal of Machine Learning Research}, $\bm{8}$, 1393-1429.

\noindent Chu, W., $\&$ Keerthi, S. S.(2005). New approaches to support vector ordinal regression. \textit{In Proceedings of the 22nd international conference on Machine learning}, 145-152.

\noindent Frank, E., $\&$ Hall, M.(2001). A simple approach to ordinal classification. \textit{ECML'01: Proceedings of the 12th European Conference on Machine Learning}, 145-156.

\noindent Jain, A. P., $\&$ Dandannavar, P.(2016). Application of machine learning techniques to sentiment analysis. \textit{In 2016 2nd International Conference on Applied and Theoretical Computing and Communication Technology (iCATccT)}, 628-632.

\noindent Koren, Y., $\&$ Sill, J.(2011). Ordrec: an ordinal model for predicting personalized item rating distributions. \textit{In Proceedings of the fifth ACM conference on Recommender systems}, 117-124.


\newpage 


\noindent Opitz, J., $\&$ Burst, S.(2019). Macro f1 and macro f1. \textit{arXiv preprint arXiv:1911.03347}.

\noindent Rennie, J. D., $\&$ Srebro, N.(2005). Loss functions for preference levels: Regression with discrete ordered labels. \textit{In Proceedings of the IJCAI multidisciplinary workshop on advances in preference handling}, $\bm{1}$.

\noindent Saad, S. E., $\&$ Yang, J.(2019). Twitter sentiment analysis based on ordinal regression. \textit{IEEE Access}, $\bm{7}$, 163677-163685.

\noindent Jing, L. P., Huang, H. K., $\&$ Shi, H. B.(2002). Improved feature selection approach TFIDF in text mining. \textit{In Proceedings. International Conference on Machine Learning and Cybernetics, $\bm{2}$}, 944-946.

\noindent Joulin, A., Grave, E., $\&$ Dandannavar, P.(2016). Bag of tricks for efficient text classification. \textit{arXiv preprint arXiv:1607.01759}.

\noindent Vargas, V. M., Gutiérrez, P. A., $\&$  Hervás-Martínez, C.(2020). Cumulative link models for deep ordinal classification. \textit{Neurocomputing, }$\bm{401}$, 48-58.

\noindent Bojanowski, P., Grave, E., Joulin, A., $\&$ Mikolov, T.(2017). Enriching word vectors with subword information. \textit{Transactions of the Association for Computational Linguistics}, $\bm{5}$, 628-632.

\noindent Liu, C., Li, Y., Ping Li, $\&$  Fei, H.(2019). Deep Skip-Gram Networks for Text Classification. \textit{In Proceedings of the 2019 SIAM International Conference on Data Mining}, 145-153.

\noindent Mikolov, T., Sutskever, I., Chen, K., Corrado, G., $\&$ Dean, J.(2013). Distributed representations of words and phrases and their compositionality. \textit{arXiv preprint arXiv:1310.4546}.





\newpage
\thispagestyle{empty}


%\end{document}

%\bibliographystyle{abbrv}                 % 指定style檔
%\bibliography{myRef}                      % 用的.bib檔名稱

%%%%%%%%%%%%%%%%%%% Appendix %%%%%%%%%%%%%%%%%%%%%%%%

%\appendix                        
%\addcontentsline{toc}{chapter}{附錄A~~~~分析方法I下之通過率}% 將"附錄"的頁碼加到目錄
%\include{appendixA}        % 附錄1
%\addcontentsline{toc}{chapter}{附錄B~~~~分析方法II下之通過率}% 將"附錄"的頁碼加到目錄
%\include{appendixB}        % 附錄2
%\addcontentsline{toc}{chapter}{附錄C~~~~分析方法III下之通過率}% 將"附錄"的頁碼加到目錄
%\include{appendixC}        % 附錄3

%%%%%%%%%%%%%%%%%%%%%%%%%%%%%%%%%%%%%%%%%%%%%%%%%%%%%%%%%%


\end{document}
